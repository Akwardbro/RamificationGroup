
\begin{theorem}\label{ceil_nonpos}
        \leanok
        \lean{ceil_nonpos}
        \uses{ValRingHom.comp}
                No documentation.
    \end{theorem}

\begin{proof}
    \leanok
\end{proof}

\begin{theorem}\label{HerbrandFunction.phiDeriv_eq_one_of_le_zero}
        \lean{HerbrandFunction.phiDeriv_eq_one_of_le_zero}
        \uses{ceil_nonpos,lowerRamificationGroup,lowerRamificationGroup.antitone,RamificationGroup}
                No documentation.
    \end{theorem}

\begin{theorem}\label{HerbrandFunction.phi_eq_self_of_le_zero}
        \lean{HerbrandFunction.phi_eq_self_of_le_zero}
        \uses{ceil_nonpos,HerbrandFunction.phiDeriv_eq_one_of_le_zero}
                No documentation.
    \end{theorem}

\begin{theorem}\label{HerbrandFunction.phiDeriv_pos}
        \lean{HerbrandFunction.phiDeriv_pos}
                No documentation.
    \end{theorem}

\begin{theorem}\label{HerbrandFunction.phiDeriv_eq_ceil}
        \leanok
        \lean{HerbrandFunction.phiDeriv_eq_ceil}
                No documentation.
    \end{theorem}

\begin{proof}
    \leanok
\end{proof}

\begin{theorem}\label{HerbrandFunction.phi_pos_of_pos}
        \lean{HerbrandFunction.phi_pos_of_pos}
        \uses{HerbrandFunction.phiDeriv_pos}
                No documentation.
    \end{theorem}

\begin{theorem}\label{HerbrandFunction.phi_pos_gt_nonpos}
        \lean{HerbrandFunction.phi_pos_gt_nonpos}
        \uses{HerbrandFunction.phi_eq_self_of_le_zero,HerbrandFunction.phi_pos_of_pos}
                No documentation.
    \end{theorem}

\begin{theorem}\label{HerbrandFunction.phi_of_pos_of_le_one}
        \leanok
        \lean{HerbrandFunction.phi_of_pos_of_le_one}
                No documentation.
    \end{theorem}

\begin{proof}
    \leanok
\end{proof}

\begin{theorem}\label{HerbrandFunction.Finset.sum_Icc_sub_sum_Icc}
        \lean{HerbrandFunction.Finset.sum_Icc_sub_sum_Icc}
                No documentation.
    \end{theorem}

\begin{theorem}\label{HerbrandFunction.phi_strictMono_of_gt_one}
        \lean{HerbrandFunction.phi_strictMono_of_gt_one}
        \uses{HerbrandFunction.phiDeriv_pos,HerbrandFunction.phiDeriv_eq_ceil,HerbrandFunction.phi_strictMono,HerbrandFunction.ceil_eq_floor_add_one_iff,HerbrandFunction.Int.eq_of_ge_of_lt_add_one,HerbrandFunction.phi_eq_self_of_le_neg_one,ValAlgEquiv.symm,ValAlgEquiv.trans,ExtDVR.3}
                No documentation.
    \end{theorem}

\begin{theorem}\label{HerbrandFunction.phi_strictMono}
        \lean{HerbrandFunction.phi_strictMono}
        \uses{HerbrandFunction.phi_mono_int,HerbrandFunction.phi_mono_iff,phi_mono}
                No documentation.
    \end{theorem}

\begin{theorem}\label{HerbrandFunction.phi_inf}
        \lean{HerbrandFunction.phi_inf}
                No documentation.
    \end{theorem}

\begin{theorem}\label{HerbrandFunction.exist_aux}
        \lean{HerbrandFunction.exist_aux}
                No documentation.
    \end{theorem}

\begin{theorem}\label{HerbrandFunction.phi_Bijective}
        \lean{HerbrandFunction.phi_Bijective}
        \uses{HerbrandFunction.phi_strictMono,HerbrandFunction.exist_aux,ValRingHom.comp}
                No documentation.
    \end{theorem}

\begin{theorem}\label{HerbrandFunction.psi_bij}
        \lean{HerbrandFunction.psi_bij}
        \uses{HerbrandFunction.phi_Bijective,psi_bij,HerbrandFunction.phi_bij,injective,surjective}
                No documentation.
    \end{theorem}

\begin{theorem}\label{HerbrandFunction.phi_zero_eq_zero}
        \leanok
        \lean{HerbrandFunction.phi_zero_eq_zero}
        \uses{ValRingHom.comp}
                No documentation.
    \end{theorem}

\begin{proof}
    \leanok
\end{proof}

\begin{theorem}\label{HerbrandFunction.leftInverse_phi_psi}
        \lean{HerbrandFunction.leftInverse_phi_psi}
        \uses{HerbrandFunction.phi_Bijective,HerbrandFunction.phi_bij,surjective}
                No documentation.
    \end{theorem}

\begin{theorem}\label{HerbrandFunction.phi_psi_eq_self}
        \lean{HerbrandFunction.phi_psi_eq_self}
        \uses{HerbrandFunction.leftInverse_phi_psi}
                No documentation.
    \end{theorem}

\begin{theorem}\label{HerbrandFunction.psi_eq_self_of_le_neg_one}
        \lean{HerbrandFunction.psi_eq_self_of_le_neg_one}
        \uses{HerbrandFunction.phi_eq_self_of_le_zero,HerbrandFunction.phi_Bijective,HerbrandFunction.phi_psi_eq_self,HerbrandFunction.phi_bij,HerbrandFunction.phi_eq_self_of_le_neg_one,injective,ValRingHom.comp,ExtDVR.3}
                No documentation.
    \end{theorem}

\begin{theorem}\label{HerbrandFunction.Ramification_Group_Disjoint}
        \lean{HerbrandFunction.Ramification_Group_Disjoint}
        \uses{lowerRamificationGroup,lowerRamificationGroup.antitone,RamificationGroup,ValAlgEquiv.symm}
                No documentation.
    \end{theorem}

\begin{theorem}\label{HerbrandFunction.Ramification_Group_pairwiseDisjoint}
        \lean{HerbrandFunction.Ramification_Group_pairwiseDisjoint}
                No documentation.
    \end{theorem}

\begin{theorem}\label{HerbrandFunction.x_not_in_aux}
        \lean{HerbrandFunction.x_not_in_aux}
        \uses{ValAlgEquiv.refl}
                No documentation.
    \end{theorem}

\begin{theorem}\label{HerbrandFunction.x_in_G_n}
        \lean{HerbrandFunction.x_in_G_n}
        \uses{ValAlgEquiv.refl}
                No documentation.
    \end{theorem}

\begin{theorem}\label{HerbrandFunction.mem_all_lowerRamificationGroup_iff}
        \lean{HerbrandFunction.mem_all_lowerRamificationGroup_iff}
        \uses{HerbrandFunction.x_in_G_n,lowerRamificationGroup,lowerIndex_refl,lowerIndex_eq_top_iff_eq_refl,mem_lowerRamificationGroup_iff,eq_top,lowerIndex,RamificationGroup,ValAlgEquiv.refl}
                No documentation.
    \end{theorem}

\begin{theorem}\label{HerbrandFunction.m_lt_n_of_in_G_m_of_notin_G_n}
        \lean{HerbrandFunction.m_lt_n_of_in_G_m_of_notin_G_n}
        \uses{lowerRamificationGroup,lowerRamificationGroup.antitone,RamificationGroup}
                No documentation.
    \end{theorem}

\begin{theorem}\label{HerbrandFunction.G_n_or_G_lt_n}
        \lean{HerbrandFunction.G_n_or_G_lt_n}
        \uses{HerbrandFunction.x_in_G_n,HerbrandFunction.mem_all_lowerRamificationGroup_iff,HerbrandFunction.m_lt_n_of_in_G_m_of_notin_G_n,lowerRamificationGroup,RamificationGroup,ValAlgEquiv.refl}
                No documentation.
    \end{theorem}

\begin{theorem}\label{HerbrandFunction.Raimification_Group_split}
        \lean{HerbrandFunction.Raimification_Group_split}
        \uses{HerbrandFunction.Ramification_Group_pairwiseDisjoint,HerbrandFunction.G_n_or_G_lt_n,HerbrandFunction.Sum_Trunc_lower_Index_of_G_n,HerbrandFunction.Sum_Trunc_lower_Index_of_diff_G,ext,ValAlgEquiv.refl}
                No documentation.
    \end{theorem}

\begin{theorem}\label{HerbrandFunction.relindex_aux}
        \lean{HerbrandFunction.relindex_aux}
                No documentation.
    \end{theorem}

\begin{theorem}\label{HerbrandFunction.phi_eq_sum_card}
        \lean{HerbrandFunction.phi_eq_sum_card}
        \uses{HerbrandFunction.relindex_aux,lowerRamificationGroup,lowerRamificationGroup.antitone,RamificationGroup,ext,ValAlgEquiv.symm}
                No documentation.
    \end{theorem}

\begin{theorem}\label{HerbrandFunction.insert_Icc_left}
        \leanok
        \lean{HerbrandFunction.insert_Icc_left}
        \uses{ext,ValAlgEquiv.symm,ValAlgEquiv.trans}
                No documentation.
    \end{theorem}

\begin{proof}
    \leanok
\end{proof}

\begin{theorem}\label{HerbrandFunction.insert_Icc_right}
        \leanok
        \lean{HerbrandFunction.insert_Icc_right}
        \uses{ValAlgEquiv.symm}
                No documentation.
    \end{theorem}

\begin{proof}
    \leanok
\end{proof}

\begin{theorem}\label{HerbrandFunction.sum_insert_left_aux}
        \leanok
        \lean{HerbrandFunction.sum_insert_left_aux}
        \uses{HerbrandFunction.insert_Icc_left}
                No documentation.
    \end{theorem}

\begin{proof}
    \leanok
\end{proof}

\begin{theorem}\label{HerbrandFunction.sum_insert_left_aux'}
        \leanok
        \lean{HerbrandFunction.sum_insert_left_aux'}
        \uses{HerbrandFunction.insert_Icc_left,HerbrandFunction.sum_insert_left_aux}
                No documentation.
    \end{theorem}

\begin{proof}
    \leanok
\end{proof}

\begin{theorem}\label{HerbrandFunction.sum_insert_right_aux'}
        \leanok
        \lean{HerbrandFunction.sum_insert_right_aux'}
        \uses{HerbrandFunction.insert_Icc_right}
                No documentation.
    \end{theorem}

\begin{proof}
    \leanok
\end{proof}

\begin{theorem}\label{HerbrandFunction.sum_sub_aux}
        \lean{HerbrandFunction.sum_sub_aux}
        \uses{HerbrandFunction.sum_insert_left_aux,HerbrandFunction.sum_insert_left_aux',HerbrandFunction.sum_insert_right_aux',lowerRamificationGroup,RamificationGroup}
                No documentation.
    \end{theorem}

\begin{theorem}\label{HerbrandFunction.truncatedLowerindex_eq_if}
        \lean{HerbrandFunction.truncatedLowerindex_eq_if}
                No documentation.
    \end{theorem}

\begin{theorem}\label{HerbrandFunction.sum_of_diff_aux}
        \lean{HerbrandFunction.sum_of_diff_aux}
        \uses{HerbrandFunction.truncatedLowerindex_eq_if,lowerRamificationGroup,lowerRamificationGroup.antitone,RamificationGroup}
                No documentation.
    \end{theorem}

\begin{theorem}\label{HerbrandFunction.truncatedLowerindex_eq_of_lt}
        \lean{HerbrandFunction.truncatedLowerindex_eq_of_lt}
        \uses{lowerRamificationGroup,mem_lowerRamificationGroup_iff,RamificationGroup,ValAlgEquiv.trans}
                No documentation.
    \end{theorem}

\begin{theorem}\label{HerbrandFunction.sum_fiberwise_aux}
        \lean{HerbrandFunction.sum_fiberwise_aux}
        \uses{HerbrandFunction.Raimification_Group_split,HerbrandFunction.truncatedLowerindex_eq_of_lt,lowerRamificationGroup,lowerRamificationGroup.antitone,RamificationGroup}
                No documentation.
    \end{theorem}

\begin{theorem}\label{HerbrandFunction.cast_natCast}
        \lean{HerbrandFunction.cast_natCast}
                No documentation.
    \end{theorem}

\begin{theorem}\label{HerbrandFunction.phi_eq_sum_inf}
        \lean{HerbrandFunction.phi_eq_sum_inf}
        \uses{HerbrandFunction.G_pairwiseDisjoint,HerbrandFunction.G_split,HerbrandFunction.Sum_Trunc_lower_Index_of_G_n,HerbrandFunction.Sum_Trunc_lower_Index_of_diff_G,ValAlgEquiv.refl}
                No documentation.
    \end{theorem}

\begin{theorem}\label{HerbrandFunction.phi_eq_ofEquiv}
        \lean{HerbrandFunction.phi_eq_ofEquiv}
                No documentation.
    \end{theorem}

\begin{theorem}\label{HerbrandFunction.psi_eq_ofEquiv}
        \lean{HerbrandFunction.psi_eq_ofEquiv}
                No documentation.
    \end{theorem}

\begin{definition}\label{lowerRamificationGroup}
        \lean{lowerRamificationGroup}
        \uses{Valuation.IsEquiv_comap_symm,Valued.decompositionGroup,RamificationGroup,Valued.zero_le,IsValExtension.val_map_le_one_iff,ValRingHom.comp,ValAlgEquiv.symm,ValAlgEquiv.one_apply}
                No documentation.
    \end{definition}

\begin{theorem}\label{lowerRamificationGroup.antitone}
        \lean{lowerRamificationGroup.antitone}
        \uses{lowerRamificationGroup,RamificationGroup,ValAlgEquiv.trans}
                No documentation.
    \end{theorem}

\begin{theorem}\label{autCongr_mem_lowerRamificationGroup_iff}
        \lean{autCongr_mem_lowerRamificationGroup_iff}
        \uses{lowerRamificationGroup,mem_lowerRamificationGroup_iff,lowerIndex,RamificationGroup,ValRingHom.id,ValRingHom.comp,ValAlgEquiv.trans,ValAlgEquiv.trans_apply}
                No documentation.
    \end{theorem}

\begin{theorem}\label{lowerIndex_refl}
        \leanok
        \lean{lowerIndex_refl}
        \uses{lowerIndex,ValRingHom.id,ValAlgEquiv.refl,ValAlgEquiv.coe_refl}
                `This should be add to correct place`
    \end{theorem}

\begin{proof}
    \leanok
\end{proof}

\begin{theorem}\label{truncatedLowerIndex_refl}
        \leanok
        \lean{truncatedLowerIndex_refl}
        \uses{lowerIndex_refl,lowerIndex,ValAlgEquiv.refl}
                No documentation.
    \end{theorem}

\begin{proof}
    \leanok
\end{proof}

\begin{theorem}\label{lowerIndex_ne_refl_of_FG}
        \lean{lowerIndex_ne_refl_of_FG}
        \uses{lowerIndex_ne_refl,lowerIndex,ext,ValRingHom.id,ValAlgEquiv.refl,ValAlgEquiv.coe_refl}
                `h` should be `𝒪[L] is finite over 𝒪[K]`
    \end{theorem}

\begin{theorem}\label{lowerIndex_ne_one}
        \leanok
        \lean{lowerIndex_ne_one}
        \uses{Valued.decompositionGroup,Valued.decompositionGroup_one,lowerIndex,IsValExtension.val_map_le_one_iff,ext,ValRingHom.id,ValRingHom.comp,ValAlgEquiv.refl,ValAlgEquiv.coe_refl}
                No documentation.
    \end{theorem}

\begin{proof}
    \leanok
\end{proof}

\begin{theorem}\label{lowerIndex_eq_top_iff_eq_refl}
        \leanok
        \lean{lowerIndex_eq_top_iff_eq_refl}
        \uses{lowerIndex_ne_one,Valued.decompositionGroup,eq_top,lowerIndex,ValRingHom.comp,ValAlgEquiv.refl,ValAlgEquiv.one_apply}
                No documentation.
    \end{theorem}

\begin{proof}
    \leanok
\end{proof}

\begin{theorem}\label{mem_lowerRamificationGroup_iff}
        \lean{mem_lowerRamificationGroup_iff}
        \uses{lowerRamificationGroup,lowerIndex,RamificationGroup}
                No documentation.
    \end{theorem}

\begin{theorem}\label{mem_lowerRamificationGroup_of_le_truncatedLowerIndex_sub_one}
        \lean{mem_lowerRamificationGroup_of_le_truncatedLowerIndex_sub_one}
        \uses{lowerRamificationGroup,mem_lowerRamificationGroup_iff,RamificationGroup}
                No documentation.
    \end{theorem}

\begin{theorem}\label{le_truncatedLowerIndex_sub_one_iff_mem_lowerRamificationGroup}
        \lean{le_truncatedLowerIndex_sub_one_iff_mem_lowerRamificationGroup}
        \uses{lowerRamificationGroup,mem_lowerRamificationGroup_iff,mem_lowerRamificationGroup_of_le_truncatedLowerIndex_sub_one,lowerIndex,RamificationGroup}
                No documentation.
    \end{theorem}

\begin{theorem}\label{lowerIndex_restrictScalars}
        \leanok
        \lean{lowerIndex_restrictScalars}
        \uses{lowerIndex}
                No documentation.
    \end{theorem}

\begin{proof}
    \leanok
\end{proof}

\begin{theorem}\label{truncatedLowerIndex_restrictScalars}
        \leanok
        \lean{truncatedLowerIndex_restrictScalars}
                No documentation.
    \end{theorem}

\begin{proof}
    \leanok
\end{proof}

\begin{theorem}\label{lowerRamificationGroup_restrictScalars}
        \lean{lowerRamificationGroup_restrictScalars}
        \uses{lowerRamificationGroup,RamificationGroup,ValRingHom.comp,AlgEquiv.restrictScalarsHom}
                No documentation.
    \end{theorem}

\begin{theorem}\label{lowerRamificationGroup_eq_decompositionGroup}
        \lean{lowerRamificationGroup_eq_decompositionGroup}
        \uses{lowerRamificationGroup,Valued.decompositionGroup,eq_decompositionGroup,eq_top,RamificationGroup,IsValExtension.val_map_le_one_iff,ext,ValRingHom.comp}
                No documentation.
    \end{theorem}

\begin{theorem}\label{lowerRamificationGroup_eq_top}
        \lean{lowerRamificationGroup_eq_top}
        \uses{lowerRamificationGroup,lowerRamificationGroup_eq_decompositionGroup,Valued.decompositionGroup,Valued.decompositionGroup_eq_top,eq_decompositionGroup,eq_top,RamificationGroup,ValRingHom.comp,Valuation.Nontrivial,IsValExtension}
                No documentation.
    \end{theorem}

\begin{theorem}\label{AlgEquiv.val_map_powerBasis_sub_ne_zero}
        \leanok
        \lean{AlgEquiv.val_map_powerBasis_sub_ne_zero}
        \uses{ext,ValRingHom.id,ValAlgEquiv.refl,ValAlgEquiv.coe_refl,PowerBasis.algEquiv_ext,Valued.AlgEquiv.restrictValuationSubring,Valued.AlgEquiv.restrictValuationSubring_apply,Valued.AlgEquiv.eq_iff_ValuationSubring}
                Can't be inferred within 20000 heartbeats.
    \end{theorem}

\begin{proof}
    \leanok
\end{proof}

\begin{theorem}\label{AlgEquiv.val_map_sub_le_powerBasis}
        \leanok
        \lean{AlgEquiv.val_map_sub_le_powerBasis}
        \uses{AlgEquiv.val_map_powerBasis_sub_ne_zero,ext,ValRingHom.id,ValAlgEquiv.refl,ValAlgEquiv.coe_refl,Polynomial.taylor_order_zero_apply_aeval,Valued.AlgEquiv.restrictValuationSubring,Valued.AlgEquiv.restrictValuationSubring_apply}
                No documentation.
    \end{theorem}

\begin{proof}
    \leanok
\end{proof}

\begin{theorem}\label{AlgEquiv.iSup_val_map_sub_eq_powerBasis}
        \leanok
        \lean{AlgEquiv.iSup_val_map_sub_eq_powerBasis}
        \uses{AlgEquiv.val_map_sub_le_powerBasis,ValAlgEquiv.symm}
                No documentation.
    \end{theorem}

\begin{proof}
    \leanok
\end{proof}

\begin{theorem}\label{lowerIndex_of_powerBasis}
        \leanok
        \lean{lowerIndex_of_powerBasis}
        \uses{lowerIndex_refl,AlgEquiv.val_map_powerBasis_sub_ne_zero,AlgEquiv.iSup_val_map_sub_eq_powerBasis,lowerIndex,ValAlgEquiv.refl}
                Should I `open Classical`?
    \end{theorem}

\begin{proof}
    \leanok
\end{proof}

\begin{theorem}\label{lowerIndex_ne_refl}
        \leanok
        \lean{lowerIndex_ne_refl}
        \uses{lowerIndex_ne_one,Valued.decompositionGroup,Valued.decompositionGroup_one,Valued.decompositionGroup_eq_top,eq_top,lowerIndex,ext,ValRingHom.comp,ValAlgEquiv.refl}
                No documentation.
    \end{theorem}

\begin{proof}
    \leanok
\end{proof}

\begin{theorem}\label{aux1}
        \lean{aux1}
        \uses{lowerRamificationGroup,le_truncatedLowerIndex_sub_one_iff_mem_lowerRamificationGroup,RamificationGroup,ValAlgEquiv.refl}
                No documentation.
    \end{theorem}

\begin{theorem}\label{aux0}
        \lean{aux0}
        \uses{ValAlgEquiv.refl}
                No documentation.
    \end{theorem}

\begin{theorem}\label{exist_lowerRamificationGroup_eq_bot}
        \lean{exist_lowerRamificationGroup_eq_bot}
        \uses{lowerRamificationGroup,aux1,RamificationGroup}
                No documentation.
    \end{theorem}

\begin{theorem}\label{Valuation.IsEquiv_comap_symm}
        \leanok
        \lean{Valuation.IsEquiv_comap_symm}
        \uses{ValAlgEquiv.symm}
                No documentation.
    \end{theorem}

\begin{proof}
    \leanok
\end{proof}

\begin{definition}\label{Valued.decompositionGroup}
        \leanok
        \lean{Valued.decompositionGroup}
        \uses{Valuation.IsEquiv_comap_symm,ValRingHom.comp,ValAlgEquiv.refl,ValAlgEquiv.symm}
                No documentation.
    \end{definition}

\begin{theorem}\label{Valued.decompositionGroup_one}
        \leanok
        \lean{Valued.decompositionGroup_one}
        \uses{Valued.decompositionGroup,eq_top,ValRingHom.comp,ValAlgEquiv.refl}
                No documentation.
    \end{theorem}

\begin{proof}
    \leanok
\end{proof}

\begin{theorem}\label{Valued.decompositionGroup_eq_top}
        \leanok
        \lean{Valued.decompositionGroup_eq_top}
        \uses{Valued.decompositionGroup,eq_top,ValAlgEquiv.algEquiv_preserve_val,ValRingHom.comp,DiscreteValuation.algEquiv_preserve_val_of_complete,IsValExtension}
                No documentation.
    \end{theorem}

\begin{proof}
    \leanok
\end{proof}

\begin{definition}\label{upperRamificationGroup_aux}
        \lean{upperRamificationGroup_aux}
        \uses{lowerRamificationGroup,upperRamificationGroup,RamificationGroup}
                No documentation.
    \end{definition}

\begin{theorem}\label{autCongr_mem_upperRamificationGroup_aux_iff}
        \lean{autCongr_mem_upperRamificationGroup_aux_iff}
        \uses{HerbrandFunction.psi_eq_ofEquiv,lowerRamificationGroup,autCongr_mem_lowerRamificationGroup_iff,mem_lowerRamificationGroup_iff,upperRamificationGroup_aux,upperRamificationGroup,RamificationGroup,ValAlgEquiv.symm,IsValExtension}
                No documentation.
    \end{theorem}

\begin{theorem}\label{preimage_singleton_nonempty}
        \leanok
        \lean{preimage_singleton_nonempty}
        \uses{surjective}
                No documentation.
    \end{theorem}

\begin{proof}
    \leanok
\end{proof}

\begin{definition}\label{HerbrandFunction.truncatedJ}
        \leanok
        \lean{HerbrandFunction.truncatedJ}
        \uses{preimage_singleton_nonempty}
                No documentation.
    \end{definition}

\begin{theorem}\label{exist_truncatedLowerIndex_eq_truncatedJ}
        \lean{exist_truncatedLowerIndex_eq_truncatedJ}
        \uses{HerbrandFunction.truncatedJ}
                No documentation.
    \end{theorem}

\begin{theorem}\label{phi_truncatedJ_sub_one}
        \lean{phi_truncatedJ_sub_one}
                No documentation.
    \end{theorem}

\begin{theorem}\label{mem_lowerRamificationGroup_of_le_truncatedJ_sub_one}
        \lean{mem_lowerRamificationGroup_of_le_truncatedJ_sub_one}
        \uses{lowerRamificationGroup,mem_lowerRamificationGroup_of_le_truncatedLowerIndex_sub_one,exist_truncatedLowerIndex_eq_truncatedJ,RamificationGroup}
                No documentation.
    \end{theorem}

\begin{theorem}\label{le_truncatedJ_sub_one_iff_mem_lowerRamificationGroup}
        \lean{le_truncatedJ_sub_one_iff_mem_lowerRamificationGroup}
        \uses{lowerRamificationGroup,le_truncatedLowerIndex_sub_one_iff_mem_lowerRamificationGroup,exist_truncatedLowerIndex_eq_truncatedJ,mem_lowerRamificationGroup_of_le_truncatedJ_sub_one,RamificationGroup,ext,IsValExtension}
                No documentation.
    \end{theorem}

\begin{theorem}\label{HerbrandFunction.phi_comp_of_isValExtension'}
        \lean{HerbrandFunction.phi_comp_of_isValExtension'}
        \uses{HerbrandFunction.phi_comp_of_isValExtension,ValRingHom.comp,ValAlgHom.comp_apply}
                No documentation.
    \end{theorem}

\begin{theorem}\label{HerbrandFunction.phi_comp_of_isValExtension}
        \lean{HerbrandFunction.phi_comp_of_isValExtension}
        \uses{HerbrandFunction.phi_comp_of_isValExtension',ext,ValRingHom.comp}
                No documentation.
    \end{theorem}

\begin{theorem}\label{HerbrandFunction.Function.comp_left_cancel}
        \lean{HerbrandFunction.Function.comp_left_cancel}
        \uses{ValRingHom.comp}
                No documentation.
    \end{theorem}

\begin{theorem}\label{HerbrandFunction.psi_comp_of_isValExtension}
        \lean{HerbrandFunction.psi_comp_of_isValExtension}
        \uses{HerbrandFunction.phi_Bijective,HerbrandFunction.phi_comp_of_isValExtension,HerbrandFunction.Function.comp_left_cancel,injective,ValRingHom.id,ValRingHom.comp,ValAlgHom.id_comp}
                No documentation.
    \end{theorem}

\begin{theorem}\label{HerbrandFunction.psi_comp_of_isValExtension'}
        \lean{HerbrandFunction.psi_comp_of_isValExtension'}
        \uses{HerbrandFunction.psi_comp_of_isValExtension,ValRingHom.comp}
                No documentation.
    \end{theorem}

\begin{theorem}\label{herbrand}
        \lean{herbrand}
        \uses{HerbrandFunction.phi_strictMono,lowerRamificationGroup,le_truncatedLowerIndex_sub_one_iff_mem_lowerRamificationGroup,phi_truncatedJ_sub_one,le_truncatedJ_sub_one_iff_mem_lowerRamificationGroup,RamificationGroup,ext,ValAlgEquiv.symm}
                No documentation.
    \end{theorem}

\begin{theorem}\label{herbrand'}
        \lean{herbrand'}
        \uses{HerbrandFunction.phi_psi_eq_self,HerbrandFunction.psi_comp_of_isValExtension,herbrand,ValRingHom.comp,ValAlgHom.comp_apply}
                No documentation.
    \end{theorem}

\begin{theorem}\label{UpperRamificationGroup_aux.eq_decompositionGroup}
        \lean{UpperRamificationGroup_aux.eq_decompositionGroup}
        \uses{HerbrandFunction.psi_eq_self_of_le_neg_one,lowerRamificationGroup,lowerRamificationGroup_eq_decompositionGroup,Valued.decompositionGroup,upperRamificationGroup_aux,upperRamificationGroup,eq_decompositionGroup,RamificationGroup,ValRingHom.comp}
                No documentation.
    \end{theorem}

\begin{theorem}\label{UpperRamificationGroup_aux.eq_top}
        \lean{UpperRamificationGroup_aux.eq_top}
        \uses{Valued.decompositionGroup,Valued.decompositionGroup_eq_top,UpperRamificationGroup_aux.eq_decompositionGroup,eq_decompositionGroup,eq_top,RamificationGroup,ValRingHom.comp,IsValExtension}
                No documentation.
    \end{theorem}

\begin{theorem}\label{UpperRamificationGroup_aux.exist_eq_bot}
        \lean{UpperRamificationGroup_aux.exist_eq_bot}
        \uses{lowerRamificationGroup,exist_lowerRamificationGroup_eq_bot,upperRamificationGroup_aux,upperRamificationGroup,RamificationGroup,ValRingHom.id,ValRingHom.comp,IsValExtension}
                No documentation.
    \end{theorem}

\begin{theorem}\label{Valued.toUniformSpace_eq_of_v_eq}
        \leanok
        \lean{Valued.toUniformSpace_eq_of_v_eq}
        \uses{ext}
                No documentation.
    \end{theorem}

\begin{proof}
    \leanok
\end{proof}

\begin{definition}\label{completeSpaceIsValExtension}
        \leanok
        \lean{completeSpaceIsValExtension}
        \uses{Valued.toUniformSpace_eq_of_v_eq,ext,ValRingHom.comp,isEquiv_iff_eq,extension_valuation_equiv_extendedValuation_of_discrete
,IsValExtension}
                No documentation.
    \end{definition}

\begin{definition}\label{upperRamificationGroup}
        \lean{upperRamificationGroup}
        \uses{upperRamificationGroup_aux,RamificationGroup,IsValExtension}
                No documentation.
    \end{definition}

\begin{theorem}\label{restrictNormal_eq_self}
        \leanok
        \lean{restrictNormal_eq_self}
        \uses{ext}
                No documentation.
    \end{theorem}

\begin{proof}
    \leanok
\end{proof}

\begin{theorem}\label{restrictNormal_restrictNormal}
        \lean{restrictNormal_restrictNormal}
                No documentation.
    \end{theorem}

\begin{theorem}\label{eq_UpperRamificationGroup_aux}
        \lean{eq_UpperRamificationGroup_aux}
        \uses{upperRamificationGroup_aux,herbrand,herbrand',upperRamificationGroup,RamificationGroup,ext,ValRingHom.id,IsValExtension}
                No documentation.
    \end{theorem}

\begin{theorem}\label{mem_iff_mem_UpperRamificationGroup_aux}
        \lean{mem_iff_mem_UpperRamificationGroup_aux}
        \uses{upperRamificationGroup_aux,upperRamificationGroup,mem_iff,RamificationGroup,ext,ValRingHom.comp,IsValExtension}
                No documentation.
    \end{theorem}

\begin{theorem}\label{map_restrictNormalHom}
        \lean{map_restrictNormalHom}
        \uses{upperRamificationGroup_aux,upperRamificationGroup,mem_iff_mem_UpperRamificationGroup_aux,mem_iff,RamificationGroup,ext}
                No documentation.
    \end{theorem}

\begin{theorem}\label{mem_iff}
        \lean{mem_iff}
        \uses{eq_UpperRamificationGroup_aux,RamificationGroup}
                No documentation.
    \end{theorem}

\begin{theorem}\label{autCongr_mem_upperRamificationGroup_iff}
        \lean{autCongr_mem_upperRamificationGroup_iff}
        \uses{upperRamificationGroup,RamificationGroup}
                No documentation.
    \end{theorem}

\begin{theorem}\label{eq_decompositionGroup}
        \lean{eq_decompositionGroup}
        \uses{Valued.decompositionGroup,UpperRamificationGroup_aux.eq_decompositionGroup,eq_UpperRamificationGroup_aux,RamificationGroup,ValRingHom.comp,IsValExtension}
                No documentation.
    \end{theorem}

\begin{theorem}\label{eq_top}
        \lean{eq_top}
        \uses{UpperRamificationGroup_aux.eq_top,eq_UpperRamificationGroup_aux,RamificationGroup,IsValExtension}
                No documentation.
    \end{theorem}

\begin{theorem}\label{UpperRamificationGroup.inf_eq_bot}
        \lean{UpperRamificationGroup.inf_eq_bot}
        \uses{ext}
                No documentation.
    \end{theorem}

\begin{theorem}\label{UpperRamificationGroup.isOpen}
        \lean{UpperRamificationGroup.isOpen}
                No documentation.
    \end{theorem}

\begin{definition}\label{UpperRamificationGroup.basis}
        \lean{UpperRamificationGroup.basis}
        \uses{upperRamificationGroup,RamificationGroup}
                No documentation.
    \end{definition}

\begin{definition}\label{DiscreteValuation.ofInt}
        \leanok
        \lean{DiscreteValuation.ofInt}
        \uses{ValRingHom.id,ExtDVR.3}
                No documentation.
    \end{definition}

\begin{theorem}\label{Valuation.integral_closure_eq_integer_of_helselian}
        \lean{Valuation.integral_closure_eq_integer_of_helselian}
        \uses{Valued.valuationSubring}
                No documentation.
    \end{theorem}

\begin{theorem}\label{Valuation.integral_closure_eq_integer_of_complete_of_ext}
        \lean{Valuation.integral_closure_eq_integer_of_complete_of_ext}
        \uses{Valued.valuationSubring,ext,ValRingHom.comp}
                WARNING : not mathematically true? more conditions?
    \end{theorem}

\begin{theorem}\label{Valuation.unique_valuationSubring_of_ext}
        \lean{Valuation.unique_valuationSubring_of_ext}
        \uses{Valuation.integral_closure_eq_integer_of_complete_of_ext,unique_valuationSubring_of_ext,Valued.valuationSubring,ext,ValRingHom.comp}
                No documentation.
    \end{theorem}

\begin{theorem}\label{Valuation.unique_val_of_ext}
        \leanok
        \lean{Valuation.unique_val_of_ext}
        \uses{unique_valuationSubring_of_ext,ValAlgebra,Valued.valuationSubring,ext,ValAlgEquiv}
                No documentation.
    \end{theorem}

\begin{proof}
    \leanok
\end{proof}

\begin{theorem}\label{ValAlgEquiv.algEnd_preserve_val}
        \leanok
        \lean{ValAlgEquiv.algEnd_preserve_val}
        \uses{Valuation.unique_val_of_ext,ValAlgebra,ext,ValAlgebra.val_isEquiv_comap}
                No documentation.
    \end{theorem}

\begin{proof}
    \leanok
\end{proof}

\begin{theorem}\label{ValAlgEquiv.algEquiv_preserve_val}
        \leanok
        \lean{ValAlgEquiv.algEquiv_preserve_val}
        \uses{ValAlgEquiv.algEnd_preserve_val}
                No documentation.
    \end{theorem}

\begin{proof}
    \leanok
\end{proof}

\begin{definition}\label{ValAlgEquiv.fromAlgEquiv}
        \lean{ValAlgEquiv.fromAlgEquiv}
        \uses{ValAlgEquiv.algEquiv_preserve_val,DiscreteValuation.fromAlgEquiv,ValAlgEquiv.mk'}
                No documentation.
    \end{definition}

\begin{definition}\label{ValAlgEquiv.equivAlgEquiv}
        \lean{ValAlgEquiv.equivAlgEquiv}
        \uses{DiscreteValuation.fromAlgEquiv,DiscreteValuation.equivAlgEquiv,ValAlgEquiv}
                No documentation.
    \end{definition}

\begin{definition}\label{DiscretelyValued.extensionFiniteDimension}
        \lean{DiscretelyValued.extensionFiniteDimension}
        \uses{ext}
                No documentation.
    \end{definition}

\begin{theorem}\label{Valuation.isEquiv_of_finiteDimensional}
        \lean{Valuation.isEquiv_of_finiteDimensional}
        \uses{ValAlgebra,ext,ValAlgebra.val_isEquiv_comap,ValAlgEquiv}
                No documentation.
    \end{theorem}

\begin{definition}\label{AlgEquiv.toValAlgEquiv}
        \lean{AlgEquiv.toValAlgEquiv}
        \uses{ValAlgEquiv,toValAlgEquiv}
                No documentation.
    \end{definition}

\begin{definition}\label{ValRingHom.ramificationIndex}
        \lean{ValRingHom.ramificationIndex}
        \uses{ValRingHom}
                No documentation.
    \end{definition}

\begin{definition}\label{Valuation.leIdeal}
        \lean{Valuation.leIdeal}
        \uses{ValuationRingTopology.Valuation.leIdeal,Valued.zero_le,ValAlgEquiv.trans}
                No documentation.
    \end{definition}

\begin{theorem}\label{Valuation.leIdeal_eq_top}
        \lean{Valuation.leIdeal_eq_top}
        \uses{eq_top,ValuationRingTopology.Valuation.leIdeal,ValRingHom.id}
                No documentation.
    \end{theorem}

\begin{definition}\label{Valuation.ltIdeal}
        \lean{Valuation.ltIdeal}
        \uses{ValuationRingTopology.Valuation.ltIdeal,ValAlgEquiv.refl}
                No documentation.
    \end{definition}

\begin{theorem}\label{Valuation.ltIdeal_eq_top}
        \lean{Valuation.ltIdeal_eq_top}
        \uses{eq_top,ValuationRingTopology.Valuation.ltIdeal,ValRingHom.id}
                No documentation.
    \end{theorem}

\begin{theorem}\label{DiscreteValuation.leIdeal_eq_ltIdeal_add_one}
        \lean{DiscreteValuation.leIdeal_eq_ltIdeal_add_one}
        \uses{DiscreteValuation.ofInt}
                No documentation.
    \end{theorem}

\begin{definition}\label{ValRingHom}
        \leanok
        \lean{ValRingHom}
        \uses{ext,ValAlgebra.val_isEquiv_comap}
                `ValRingHom R S` is the type of ring homomorphisms that preserves valuation from valued ring `R` to valued ring `S`.

Please note that the definition requires `v x ≤ v y ↔ v (f x) ≤ v (f y)` instead of `v x ≤ v y → v (f x) ≤ v (f y)`. For the latter case, one can use order-preserving ring homomorphisms.

When possible, instead of parametrizing results over `(f : ValRingHom R S)`,
you should parametrize over `(F : Type*) [ValRingHomClass F R S] (f : F)`.

When you extend this structure, make sure to extend `ValRingHomClass`.
    \end{definition}

\begin{definition}\label{ValAlgebra}
        \leanok
        \lean{ValAlgebra}
        \uses{ValRingHom,ext}
                A valued algebra over a valued commutative ring `R`, is a valued ring `A` together with a ring map into the center of `A` that preserves the valuation.
    \end{definition}

\begin{definition}\label{multiplicity'}
        \leanok
        \lean{multiplicity'}
        \uses{ValRingHom.id}
                `multiplicity a b` returns the largest natural number `n` such that
  `a ^ n ∣ b`, as a `PartENat` or natural with infinity. If `∀ n, a ^ n ∣ b`,
  then it returns `⊤`
    \end{definition}

\begin{theorem}\label{finite_nat_iff'}
        \lean{finite_nat_iff'}
                No documentation.
    \end{theorem}

\begin{definition}\label{padicValNat'}
        \leanok
        \lean{padicValNat'}
                For `p ≠ 1`, the `p`-adic valuation of a natural `n ≠ 0` is the largest natural number `k` such
that `p^k` divides `z`. If `n = 0` or `p = 1`, then `padicValNat p q` defaults to `0`.
    \end{definition}

\begin{theorem}\label{finite_ideal_iff}
        \lean{finite_ideal_iff}
        \uses{ValRingHom.id,ValRingHom.comp}
                No documentation.
    \end{theorem}

\begin{theorem}\label{ord_top}
        \lean{ord_top}
        \uses{finite_ideal_iff,ValRingHom.id,ValRingHom.comp}
                No documentation.
    \end{theorem}

\begin{definition}\label{ValuationTopology.ValuationRing.setoid}
        \lean{ValuationTopology.ValuationRing.setoid}
        \uses{ValuationRingTopology.ValuationRing.setoid,ValRingHom.id}
                The topology on a valuation ring `R` is defined to be the topology associated to the preorder of divisibility.
    \end{definition}

\begin{definition}\label{ValuationTopology.ValuationRing.ValueMonoid}
        \lean{ValuationTopology.ValuationRing.ValueMonoid}
        \uses{ValuationRingTopology.ValuationRing.setoid,ValuationRingTopology.ValuationRing.ValueMonoid,ext,ValRingHom.id}
                No documentation.
    \end{definition}

\begin{definition}\label{CompleteDiscreteValuationRing}
        \leanok
        \lean{CompleteDiscreteValuationRing}
        \uses{CompleteValuationRing,ext}
                No documentation.
    \end{definition}

\begin{definition}\label{CompleteValuationField}
        \leanok
        \lean{CompleteValuationField}
        \uses{CompleteValuationRing,ext}
                No documentation.
    \end{definition}

\begin{definition}\label{CompleteDiscreteValuationField}
        \leanok
        \lean{CompleteDiscreteValuationField}
        \uses{CompleteValuationRing,ext,ValRingHom.comp}
                No documentation.
    \end{definition}

\begin{theorem}\label{sub_of_sum}
        \leanok
        \lean{sub_of_sum}
        \uses{HerbrandFunction.sub_of_sum,ext}
                No documentation.
    \end{theorem}

\begin{proof}
    \leanok
\end{proof}

\begin{theorem}\label{ceil_eq_floor_add_one_iff}
        \leanok
        \lean{ceil_eq_floor_add_one_iff}
        \uses{HerbrandFunction.ceil_eq_floor_add_one_iff}
                No documentation.
    \end{theorem}

\begin{proof}
    \leanok
\end{proof}

\begin{theorem}\label{Int.eq_of_ge_of_lt_add_one}
        \leanok
        \lean{Int.eq_of_ge_of_lt_add_one}
        \uses{HerbrandFunction.Int.eq_of_ge_of_lt_add_one,ValAlgEquiv.symm}
                No documentation.
    \end{theorem}

\begin{proof}
    \leanok
\end{proof}

\begin{theorem}\label{phi'_eq_ceil}
        \leanok
        \lean{phi'_eq_ceil}
                No documentation.
    \end{theorem}

\begin{proof}
    \leanok
\end{proof}

\begin{theorem}\label{phi'_pos}
        \lean{phi'_pos}
                No documentation.
    \end{theorem}

\begin{theorem}\label{phi'_neg_int_eq_one}
        \lean{phi'_neg_int_eq_one}
                No documentation.
    \end{theorem}

\begin{theorem}\label{phi_int_succ}
        \lean{phi_int_succ}
        \uses{HerbrandFunction.sub_of_sum,phi'_neg_int_eq_one,HerbrandFunction.phi_int_succ,ValAlgEquiv.symm}
                No documentation.
    \end{theorem}

\begin{theorem}\label{phi_mono_int}
        \lean{phi_mono_int}
        \uses{phi'_pos,HerbrandFunction.phi_int_succ,HerbrandFunction.phi_mono_int,phi_mono,ValAlgEquiv.trans}
                No documentation.
    \end{theorem}

\begin{theorem}\label{phi_mono_int'}
        \lean{phi_mono_int'}
        \uses{HerbrandFunction.phi_mono_int,HerbrandFunction.phi_mono_int',phi_mono}
                No documentation.
    \end{theorem}

\begin{theorem}\label{phi_rational_floor}
        \lean{phi_rational_floor}
        \uses{HerbrandFunction.sub_of_sum,HerbrandFunction.ceil_eq_floor_add_one_iff,phi'_eq_ceil,HerbrandFunction.phi_rational_floor,ValAlgEquiv.symm,ValAlgEquiv.trans}
                No documentation.
    \end{theorem}

\begin{theorem}\label{phi_rational_ceil}
        \lean{phi_rational_ceil}
        \uses{HerbrandFunction.sub_of_sum,HerbrandFunction.ceil_eq_floor_add_one_iff,HerbrandFunction.Int.eq_of_ge_of_lt_add_one,phi'_eq_ceil,HerbrandFunction.phi_rational_ceil,ValAlgEquiv.symm,ExtDVR.3}
                No documentation.
    \end{theorem}

\begin{theorem}\label{phi_gt_floor}
        \lean{phi_gt_floor}
        \uses{HerbrandFunction.phi_mono_int,HerbrandFunction.phi_rational_floor,HerbrandFunction.phi_gt_floor,phi_mono}
                No documentation.
    \end{theorem}

\begin{theorem}\label{phi_lt_ceil}
        \lean{phi_lt_ceil}
        \uses{HerbrandFunction.phi_mono_int,HerbrandFunction.phi_rational_ceil,HerbrandFunction.phi_lt_ceil,phi_mono}
                No documentation.
    \end{theorem}

\begin{theorem}\label{phi_mono_in_section}
        \lean{phi_mono_in_section}
        \uses{HerbrandFunction.phi_mono_int,HerbrandFunction.phi_rational_floor,HerbrandFunction.phi_mono_in_section,phi_mono}
                No documentation.
    \end{theorem}

\begin{theorem}\label{phi_mono_over_section}
        \lean{phi_mono_over_section}
        \uses{HerbrandFunction.phi_mono_int,HerbrandFunction.phi_mono_int',HerbrandFunction.phi_gt_floor,HerbrandFunction.phi_lt_ceil,HerbrandFunction.phi_mono_over_section,phi_mono,ValAlgEquiv.symm}
                No documentation.
    \end{theorem}

\begin{theorem}\label{phi_mono_iff}
        \lean{phi_mono_iff}
        \uses{HerbrandFunction.phi_mono_int,HerbrandFunction.phi_mono_in_section,HerbrandFunction.phi_mono_over_section,HerbrandFunction.phi_mono_iff,phi_mono}
                No documentation.
    \end{theorem}

\begin{theorem}\label{phi_mono}
        \lean{phi_mono}
        \uses{HerbrandFunction.phi_mono_int,HerbrandFunction.phi_mono_iff}
                No documentation.
    \end{theorem}

\begin{theorem}\label{phi_bij}
        \lean{phi_bij}
        \uses{phi_mono,HerbrandFunction.phi_bij,ValRingHom.comp}
                No documentation.
    \end{theorem}

\begin{theorem}\label{psi_bij}
        \lean{psi_bij}
        \uses{ExtDVR.3}
                No documentation.
    \end{theorem}

\begin{theorem}\label{phi_zero_eq_zero}
        \leanok
        \lean{phi_zero_eq_zero}
        \uses{HerbrandFunction.phi_zero_eq_zero,ValRingHom.comp}
                No documentation.
    \end{theorem}

\begin{proof}
    \leanok
\end{proof}

\begin{theorem}\label{psi_zero_eq_zero}
        \leanok
        \lean{psi_zero_eq_zero}
                No documentation.
    \end{theorem}

\begin{proof}
    \leanok
\end{proof}

\begin{theorem}\label{phi_inv_psi}
        \lean{phi_inv_psi}
        \uses{ValAlgebra,HerbrandFunction.phi_bij,surjective,ValRingHom.comp,ExtDVR.3}
                No documentation.
    \end{theorem}

\begin{theorem}\label{G_pairwiseDisjoint}
        \lean{G_pairwiseDisjoint}
        \uses{HerbrandFunction.G_pairwiseDisjoint}
                No documentation.
    \end{theorem}

\begin{theorem}\label{G_n_or_G_lt_n}
        \lean{G_n_or_G_lt_n}
        \uses{HerbrandFunction.G_n_or_G_lt_n}
                No documentation.
    \end{theorem}

\begin{theorem}\label{G_split}
        \lean{G_split}
        \uses{HerbrandFunction.G_n_or_G_lt_n,HerbrandFunction.G_pairwiseDisjoint,HerbrandFunction.G_split,ext}
                No documentation.
    \end{theorem}

\begin{theorem}\label{Sum_Trunc_lower_Index_of_G_n}
        \lean{Sum_Trunc_lower_Index_of_G_n}
        \uses{HerbrandFunction.Sum_Trunc_lower_Index_of_G_n,ValAlgEquiv.refl,ValAlgEquiv}
                No documentation.
    \end{theorem}

\begin{theorem}\label{Sum_Trunc_lower_Index_of_diff_G}
        \lean{Sum_Trunc_lower_Index_of_diff_G}
        \uses{HerbrandFunction.Sum_Trunc_lower_Index_of_diff_G,ValAlgEquiv.refl,ValAlgEquiv}
                No documentation.
    \end{theorem}

\begin{theorem}\label{Varphi_eq_Sum_Inf}
        \lean{Varphi_eq_Sum_Inf}
                No documentation.
    \end{theorem}

\begin{theorem}\label{HerbrandFunction.sub_of_sum}
        \leanok
        \lean{HerbrandFunction.sub_of_sum}
        \uses{ext}
                No documentation.
    \end{theorem}

\begin{proof}
    \leanok
\end{proof}

\begin{theorem}\label{HerbrandFunction.ceil_eq_floor_add_one_iff}
        \leanok
        \lean{HerbrandFunction.ceil_eq_floor_add_one_iff}
                No documentation.
    \end{theorem}

\begin{proof}
    \leanok
\end{proof}

\begin{theorem}\label{HerbrandFunction.Int.eq_of_ge_of_lt_add_one}
        \leanok
        \lean{HerbrandFunction.Int.eq_of_ge_of_lt_add_one}
        \uses{ValAlgEquiv.symm}
                No documentation.
    \end{theorem}

\begin{proof}
    \leanok
\end{proof}

\begin{theorem}\label{HerbrandFunction.phiDeriv_neg_int_eq_one}
        \lean{HerbrandFunction.phiDeriv_neg_int_eq_one}
                No documentation.
    \end{theorem}

\begin{theorem}\label{HerbrandFunction.phi_int_succ}
        \lean{HerbrandFunction.phi_int_succ}
        \uses{HerbrandFunction.sub_of_sum,HerbrandFunction.phiDeriv_neg_int_eq_one,ValAlgEquiv.symm}
                No documentation.
    \end{theorem}

\begin{theorem}\label{HerbrandFunction.phi_eq_self_of_le_neg_one}
        \lean{HerbrandFunction.phi_eq_self_of_le_neg_one}
                No documentation.
    \end{theorem}

\begin{theorem}\label{HerbrandFunction.phi_mono_int}
        \lean{HerbrandFunction.phi_mono_int}
        \uses{HerbrandFunction.phiDeriv_pos,HerbrandFunction.phi_int_succ,phi_mono,ValAlgEquiv.trans}
                No documentation.
    \end{theorem}

\begin{theorem}\label{HerbrandFunction.phi_mono_int'}
        \lean{HerbrandFunction.phi_mono_int'}
        \uses{HerbrandFunction.phi_mono_int,phi_mono}
                No documentation.
    \end{theorem}

\begin{theorem}\label{HerbrandFunction.phi_rational_floor}
        \lean{HerbrandFunction.phi_rational_floor}
        \uses{HerbrandFunction.phiDeriv_eq_ceil,HerbrandFunction.sub_of_sum,HerbrandFunction.ceil_eq_floor_add_one_iff,HerbrandFunction.phi_eq_self_of_le_neg_one,ValAlgEquiv.symm,ValAlgEquiv.trans}
                No documentation.
    \end{theorem}

\begin{theorem}\label{HerbrandFunction.phi_rational_ceil}
        \leanok
        \lean{HerbrandFunction.phi_rational_ceil}
        \uses{HerbrandFunction.sub_of_sum,HerbrandFunction.ceil_eq_floor_add_one_iff}
                No documentation.
    \end{theorem}

\begin{proof}
    \leanok
\end{proof}

\begin{theorem}\label{HerbrandFunction.phi_gt_floor}
        \lean{HerbrandFunction.phi_gt_floor}
        \uses{HerbrandFunction.phi_mono_int,HerbrandFunction.phi_rational_floor,phi_mono}
                No documentation.
    \end{theorem}

\begin{theorem}\label{HerbrandFunction.phi_lt_ceil}
        \lean{HerbrandFunction.phi_lt_ceil}
        \uses{HerbrandFunction.phi_mono_int,HerbrandFunction.phi_rational_ceil,phi_mono}
                No documentation.
    \end{theorem}

\begin{theorem}\label{HerbrandFunction.phi_mono_in_section}
        \lean{HerbrandFunction.phi_mono_in_section}
        \uses{HerbrandFunction.phi_mono_int,HerbrandFunction.phi_rational_floor,phi_mono}
                No documentation.
    \end{theorem}

\begin{theorem}\label{HerbrandFunction.phi_mono_over_section}
        \lean{HerbrandFunction.phi_mono_over_section}
        \uses{HerbrandFunction.phi_mono_int,HerbrandFunction.phi_mono_int',HerbrandFunction.phi_gt_floor,HerbrandFunction.phi_lt_ceil,phi_mono,ValAlgEquiv.symm}
                No documentation.
    \end{theorem}

\begin{theorem}\label{HerbrandFunction.phi_mono_iff}
        \lean{HerbrandFunction.phi_mono_iff}
        \uses{HerbrandFunction.phiDeriv_pos,HerbrandFunction.phi_mono_int,HerbrandFunction.phi_mono_in_section,HerbrandFunction.phi_mono_over_section,phi_mono}
                No documentation.
    \end{theorem}

\begin{theorem}\label{HerbrandFunction.phi_bij}
        \lean{HerbrandFunction.phi_bij}
        \uses{HerbrandFunction.phi_eq_self_of_le_zero,HerbrandFunction.phiDeriv_pos,HerbrandFunction.phi_pos_gt_nonpos,HerbrandFunction.phi_of_pos_of_le_one,HerbrandFunction.phi_strictMono_of_gt_one,HerbrandFunction.phi_strictMono,ValAlgEquiv.trans}
                No documentation.
    \end{theorem}

\begin{theorem}\label{HerbrandFunction.psi_zero_eq_zero}
        \lean{HerbrandFunction.psi_zero_eq_zero}
        \uses{HerbrandFunction.phi_Bijective,HerbrandFunction.phi_zero_eq_zero,HerbrandFunction.phi_bij,psi_zero_eq_zero,ValRingHom.id,ValRingHom.comp}
                No documentation.
    \end{theorem}

\begin{theorem}\label{HerbrandFunction.G_pairwiseDisjoint}
        \lean{HerbrandFunction.G_pairwiseDisjoint}
                No documentation.
    \end{theorem}

\begin{theorem}\label{HerbrandFunction.G_split}
        \lean{HerbrandFunction.G_split}
        \uses{HerbrandFunction.G_n_or_G_lt_n,HerbrandFunction.G_pairwiseDisjoint,ext}
                No documentation.
    \end{theorem}

\begin{theorem}\label{HerbrandFunction.Sum_Trunc_lower_Index_of_G_n}
        \lean{HerbrandFunction.Sum_Trunc_lower_Index_of_G_n}
        \uses{ValAlgEquiv.refl}
                No documentation.
    \end{theorem}

\begin{theorem}\label{HerbrandFunction.Sum_Trunc_lower_Index_of_diff_G}
        \lean{HerbrandFunction.Sum_Trunc_lower_Index_of_diff_G}
        \uses{ValAlgEquiv.refl}
                No documentation.
    \end{theorem}

\begin{theorem}\label{Prop18}
        \lean{Prop18}
                No documentation.
    \end{theorem}

\begin{definition}\label{aaa}
        \leanok
        \lean{aaa}
                No documentation.
    \end{definition}

\begin{theorem}\label{lowerindex_ge_iff_lowerramificationGroup}
        \lean{lowerindex_ge_iff_lowerramificationGroup}
                No documentation.
    \end{theorem}

\begin{theorem}\label{lowerindex_eq_iff_lowerramificationGroup}
        \lean{lowerindex_eq_iff_lowerramificationGroup}
                No documentation.
    \end{theorem}

\begin{theorem}\label{lowerramificationGroup_has_top}
        \lean{lowerramificationGroup_has_top}
        \uses{lowerIndex,ValRingHom.comp,ValAlgEquiv}
                No documentation.
    \end{theorem}

\begin{definition}\label{Lift_Galois_ValEquiv}
        \lean{Lift_Galois_ValEquiv}
                No documentation.
    \end{definition}

\begin{definition}\label{Galois_to_Quotient}
        \lean{Galois_to_Quotient}
        \uses{ValAlgebra,ValAlgebra.val_isEquiv_comap}
                No documentation.
    \end{definition}

\begin{theorem}\label{Varphi_With_i'}
        \lean{Varphi_With_i'}
        \uses{Lift_Galois_ValEquiv,Galois_to_Quotient,ValAlgEquiv.mk'}
                No documentation.
    \end{theorem}

\begin{theorem}\label{Herbrand_Thm}
        \lean{Herbrand_Thm}
        \uses{Lift_Galois_ValEquiv,Galois_to_Quotient,ValAlgEquiv.mk'}
                No documentation.
    \end{theorem}

\begin{theorem}\label{varphi_comp_field_ext}
        \lean{varphi_comp_field_ext}
        \uses{ext,ValRingHom.comp}
                No documentation.
    \end{theorem}

\begin{theorem}\label{psi_comp_field_ext}
        \lean{psi_comp_field_ext}
        \uses{ext,ValRingHom.comp}
                No documentation.
    \end{theorem}

\begin{definition}\label{LinearEquiv.ofSubalgebraEqTop}
        \lean{LinearEquiv.ofSubalgebraEqTop}
        \uses{eq_top}
                No documentation.
    \end{definition}

\begin{definition}\label{AlgEquiv.ofTop}
        \leanok
        \lean{AlgEquiv.ofTop}
        \uses{ValAlgEquiv.symm}
                No documentation.
    \end{definition}

\begin{theorem}\label{varphi'_pos}
        \lean{varphi'_pos}
        \uses{phi'_pos}
                No documentation.
    \end{theorem}

\begin{theorem}\label{varphi_mono_int}
        \lean{varphi_mono_int}
        \uses{HerbrandFunction.phi_mono_int,phi_mono,varphi_mono}
                No documentation.
    \end{theorem}

\begin{theorem}\label{varphi_lt_int_ceil}
        \lean{varphi_lt_int_ceil}
                No documentation.
    \end{theorem}

\begin{theorem}\label{varphi_mono_section}
        \lean{varphi_mono_section}
        \uses{phi'_pos,phi_mono,varphi'_pos,varphi_mono}
                No documentation.
    \end{theorem}

\begin{theorem}\label{varphi_mono}
        \lean{varphi_mono}
        \uses{HerbrandFunction.phi_mono_int,phi_mono,varphi_mono_int,varphi_lt_int_ceil,varphi_mono_section}
                No documentation.
    \end{theorem}

\begin{theorem}\label{varphi_bij}
        \lean{varphi_bij}
        \uses{HerbrandFunction.phi_bij}
                No documentation.
    \end{theorem}

\begin{theorem}\label{varphi_zero_eq_zero}
        \leanok
        \lean{varphi_zero_eq_zero}
        \uses{HerbrandFunction.phi_zero_eq_zero}
                No documentation.
    \end{theorem}

\begin{proof}
    \leanok
\end{proof}

\begin{theorem}\label{varphi_negone_eq_negone}
        \lean{varphi_negone_eq_negone}
        \uses{ValRingHom.comp}
                No documentation.
    \end{theorem}

\begin{definition}\label{ValuationExtension''}
        \lean{ValuationExtension''}
        \uses{PreserveValuation,ValuationExtension,ValuationExtension',ext,ValRingHom.id}
                No documentation.
    \end{definition}

\begin{definition}\label{Valuation.ramificationIndex}
        \lean{Valuation.ramificationIndex}
        \uses{ValRingHom.ramificationIndex,ValuationExtension}
                No documentation.
    \end{definition}

\begin{definition}\label{DiscreteValuation.toInt}
        \lean{DiscreteValuation.toInt}
                No documentation.
    \end{definition}

\begin{definition}\label{lowerIndex}
        \leanok
        \lean{lowerIndex}
                No documentation.
    \end{definition}

\begin{definition}\label{RamificationGroup}
        \lean{RamificationGroup}
        \uses{lowerIndex,ValuationExtension,ValRingHom.id,ValRingHom.comp}
                No documentation.
    \end{definition}

\begin{theorem}\label{lower_numbering_inf}
        \lean{lower_numbering_inf}
                No documentation.
    \end{theorem}

\begin{theorem}\label{index_subgroup}
        \lean{index_subgroup}
        \uses{ValuationExtension,ValRingHom.comp}
                No documentation.
    \end{theorem}

\begin{theorem}\label{index_quotient_group}
        \lean{index_quotient_group}
                No documentation.
    \end{theorem}

\begin{definition}\label{RingHom.liftValuationInteger}
        \lean{RingHom.liftValuationInteger}
                No documentation.
    \end{definition}

\begin{definition}\label{PreserveValuation}
        \leanok
        \lean{PreserveValuation}
        \uses{ValRingHom.id}
                No documentation.
    \end{definition}

\begin{definition}\label{ValuationExtension}
        \leanok
        \lean{ValuationExtension}
        \uses{ext,ValRingHom.id}
                No documentation.
    \end{definition}

\begin{definition}\label{ValuationExtension'}
        \lean{ValuationExtension'}
        \uses{PreserveValuation,ValuationExtension,ext,ValRingHom.id}
                No documentation.
    \end{definition}

\begin{theorem}\label{Valuation.isEquiv_of_finiteDimensional_of_valuationExtension}
        \lean{Valuation.isEquiv_of_finiteDimensional_of_valuationExtension}
        \uses{Valuation.isEquiv_of_finiteDimensional,ValuationExtension}
                No documentation.
    \end{theorem}

\begin{theorem}\label{Valuation.preserveValuation_of_finiteDimensional_algebra_hom}
        \lean{Valuation.preserveValuation_of_finiteDimensional_algebra_hom}
        \uses{PreserveValuation}
                No documentation.
    \end{theorem}

\begin{definition}\label{AlgHom.liftValuationInteger}
        \lean{AlgHom.liftValuationInteger}
                No documentation.
    \end{definition}

\begin{definition}\label{AlgEquiv.liftValuationInteger}
        \lean{AlgEquiv.liftValuationInteger}
        \uses{AlgHom.liftValuationInteger}
                No documentation.
    \end{definition}

\begin{definition}\label{AlgHom.liftResidueField}
        \lean{AlgHom.liftResidueField}
        \uses{Valued.ResidueField,ValRingHom.id,liftResidueField}
                No documentation.
    \end{definition}

\begin{definition}\label{AlgEquiv.liftResidueField}
        \lean{AlgEquiv.liftResidueField}
        \uses{ValAlgebra,Valued.ResidueField,ValRingHom.id,ValAlgEquiv,liftResidueField}
                No documentation.
    \end{definition}

\begin{definition}\label{CompleteValuationRing}
        \leanok
        \lean{CompleteValuationRing}
        \uses{ext}
                No documentation.
    \end{definition}

\begin{definition}\label{CompleteValuedField}
        \leanok
        \lean{CompleteValuedField}
        \uses{CompleteValuationRing,ValRingHom.comp}
                No documentation.
    \end{definition}

\begin{definition}\label{CompleteDiscreteValuedField}
        \leanok
        \lean{CompleteDiscreteValuedField}
        \uses{CompleteValuedField,ext}
                No documentation.
    \end{definition}

\begin{definition}\label{ValuationRingTopology.ValuationRing.setoid}
        \lean{ValuationRingTopology.ValuationRing.setoid}
        \uses{ValRingHom.id}
                The preorder of divisibility associated to a valuation ring, i.e. `a ≤ b` if there exist `c`, such that `a * c = b`.
    \end{definition}

\begin{definition}\label{ValuationRingTopology.ValuationRing.ValueMonoid}
        \lean{ValuationRingTopology.ValuationRing.ValueMonoid}
        \uses{ValuationRingTopology.ValuationRing.setoid,ValRingHom.id}
                No documentation.
    \end{definition}

\begin{definition}\label{ValuationRingTopology.Valuation.leIdeal}
        \lean{ValuationRingTopology.Valuation.leIdeal}
        \uses{ValRingHom.id}
                No documentation.
    \end{definition}

\begin{definition}\label{ValuationRingTopology.Valuation.ltIdeal}
        \lean{ValuationRingTopology.Valuation.ltIdeal}
        \uses{ValRingHom.id}
                No documentation.
    \end{definition}

\begin{definition}\label{ValuationRingTopology.Valuation.maximalIdeal}
        \lean{ValuationRingTopology.Valuation.maximalIdeal}
        \uses{ValuationRingTopology.Valuation.ltIdeal,Valued.maximalIdeal,Valued.ResidueField,ValRingHom.id}
                No documentation.
    \end{definition}

\begin{theorem}\label{Valued.le_iff_val_le}
        \lean{Valued.le_iff_val_le}
                No documentation.
    \end{theorem}

\begin{theorem}\label{Valued.lt_iff_val_lt}
        \lean{Valued.lt_iff_val_lt}
                No documentation.
    \end{theorem}

\begin{theorem}\label{Valued.le_one_iff_val_le_one}
        \lean{Valued.le_one_iff_val_le_one}
                No documentation.
    \end{theorem}

\begin{theorem}\label{Valued.lt_one_iff_val_lt_one}
        \lean{Valued.lt_one_iff_val_lt_one}
                No documentation.
    \end{theorem}

\begin{theorem}\label{Valued.zero_le}
        \lean{Valued.zero_le}
                No documentation.
    \end{theorem}

\begin{definition}\label{Valued.valuationSubring}
        \leanok
        \lean{Valued.valuationSubring}
                A `Valued` version of `Valuation.valuationSubring`, enabling the notation `𝒪[K]` for valued field `K`
    \end{definition}

\begin{theorem}\label{Valued.integer_valuation_eq}
        \leanok
        \lean{Valued.integer_valuation_eq}
        \uses{integer_valuation_eq}
                No documentation.
    \end{theorem}

\begin{proof}
    \leanok
\end{proof}

\begin{theorem}\label{Valued.integerAlgebraMap.monotone}
        \lean{Valued.integerAlgebraMap.monotone}
        \uses{integerAlgebraMap.monotone}
                No documentation.
    \end{theorem}

\begin{theorem}\label{Valued.integer_val_coe}
        \leanok
        \lean{Valued.integer_val_coe}
        \uses{integer_val_coe}
                No documentation.
    \end{theorem}

\begin{proof}
    \leanok
\end{proof}

\begin{theorem}\label{Valued.valuationSubring_val_coe}
        \leanok
        \lean{Valued.valuationSubring_val_coe}
        \uses{Valued.valuationSubring,valuationSubring_val_coe}
                No documentation.
    \end{theorem}

\begin{proof}
    \leanok
\end{proof}

\begin{theorem}\label{Valued.integer_val_le_one}
        \leanok
        \lean{Valued.integer_val_le_one}
        \uses{integer_val_le_one}
                No documentation.
    \end{theorem}

\begin{proof}
    \leanok
\end{proof}

\begin{definition}\label{Valued.maximalIdeal}
        \leanok
        \lean{Valued.maximalIdeal}
                An abbrevation for `LocalRing.maximalIdeal 𝒪[K]` of a valued field `K`, enabling the notation `𝓂[K]`
    \end{definition}

\begin{theorem}\label{Valued.maximalIdeal_eq}
        \lean{Valued.maximalIdeal_eq}
        \uses{Valued.maximalIdeal,maximalIdeal_eq}
                No documentation.
    \end{theorem}

\begin{definition}\label{Valued.ResidueField}
        \leanok
        \lean{Valued.ResidueField}
        \uses{Valued.valuationSubring,ValRingHom.id,ValAlgEquiv.mk',liftInteger}
                An abbrevation for `LocalRing.ResidueField 𝒪[K]` of a `Valued` instance, enabling the notation `𝓀[K]`
    \end{definition}

\begin{definition}\label{ValRingEquiv}
        \leanok
        \lean{ValRingEquiv}
        \uses{ext,ValAlgebra.val_isEquiv_comap}
                `ValRingEquiv R S` is the type of ring isomorphisms that preserves valuation from valued ring `R` to valued ring `S`.

When possible, instead of parametrizing results over `(f : ValRingEquiv R S)`,
you should parametrize over `(F : Type*) [ValRingEquivClass F R S] (f : F)`.

When you extend this structure, make sure to extend `ValRingEquivClass`.
    \end{definition}

\begin{definition}\label{ValRingHomClass}
        \leanok
        \lean{ValRingHomClass}
        \uses{ValRingHom,ext,ValAlgebra.val_isEquiv_comap}
                `ValHomClass F R S` asserts that `F` is a type of valuation-preserving morphisms.
    \end{definition}

\begin{definition}\label{ValRingEquivClass}
        \leanok
        \lean{ValRingEquivClass}
        \uses{ValRingEquiv,ext,ValAlgebra.val_isEquiv_comap}
                No documentation.
    \end{definition}

\begin{definition}\label{ValRingHomClass.toValRingHom}
        \leanok
        \lean{ValRingHomClass.toValRingHom}
        \uses{ValRingHom,ValRingHomClass,ValAlgebra.val_isEquiv_comap}
                Turn an element of a type `F` satisfying `ValRingHomClass F R S` into an actual
`ValRingHom`. This is declared as the default coercion from `F` to `R →+*v S`.
    \end{definition}

\begin{definition}\label{ValRingEquivClass.toValRingEquiv}
        \leanok
        \lean{ValRingEquivClass.toValRingEquiv}
        \uses{ValRingEquiv,ValRingEquivClass,ValAlgebra.val_isEquiv_comap}
                Turn an element of a type `F` satisfying `ValRingEquivClass F R S` into an actual
`ValRingEquiv`. This is declared as the default coercion from `F` to `R ≃+*v S`.
    \end{definition}

\begin{theorem}\label{ValAlgebra.val_map_le_iff}
        \leanok
        \lean{ValAlgebra.val_map_le_iff}
        \uses{IsValExtension.val_map_le_iff,valAlgebraMap}
                Any type satisfying `ValRingEquivClass` can be cast into `ValRingEquiv` via
`ValRingEquivClass.toValRingEquiv`.
    \end{theorem}

\begin{proof}
    \leanok
\end{proof}

\begin{theorem}\label{ValAlgebra.val_map_lt_iff}
        \leanok
        \lean{ValAlgebra.val_map_lt_iff}
        \uses{IsValExtension.val_map_lt_iff,valAlgebraMap}
                No documentation.
    \end{theorem}

\begin{proof}
    \leanok
\end{proof}

\begin{theorem}\label{ValAlgebra.val_map_le_one_iff}
        \leanok
        \lean{ValAlgebra.val_map_le_one_iff}
        \uses{IsValExtension.val_map_le_one_iff,valAlgebraMap}
                No documentation.
    \end{theorem}

\begin{proof}
    \leanok
\end{proof}

\begin{theorem}\label{ValAlgebra.val_map_lt_one_iff}
        \leanok
        \lean{ValAlgebra.val_map_lt_one_iff}
        \uses{IsValExtension.val_map_lt_one_iff,valAlgebraMap}
                No documentation.
    \end{theorem}

\begin{proof}
    \leanok
\end{proof}

\begin{theorem}\label{val_map_inv_le_val_iff}
        \leanok
        \lean{val_map_inv_le_val_iff}
        \uses{IsValExtension.val_map_le_iff,ValAlgEquiv.symm}
                No documentation.
    \end{theorem}

\begin{proof}
    \leanok
\end{proof}

\begin{theorem}\label{val_le_val_map_inv_iff}
        \leanok
        \lean{val_le_val_map_inv_iff}
        \uses{IsValExtension.val_map_le_iff,ValAlgEquiv.symm}
                No documentation.
    \end{theorem}

\begin{proof}
    \leanok
\end{proof}

\begin{theorem}\label{val_map_inv_lt_val_iff}
        \leanok
        \lean{val_map_inv_lt_val_iff}
        \uses{val_le_val_map_inv_iff}
                No documentation.
    \end{theorem}

\begin{proof}
    \leanok
\end{proof}

\begin{theorem}\label{val_lt_val_map_inv_iff}
        \leanok
        \lean{val_lt_val_map_inv_iff}
        \uses{ValRingHom,ValRingEquiv,ValRingHomClass,ValRingEquivClass,val_map_inv_le_val_iff,ext,injective,ValAlgebra.val_isEquiv_comap}
                No documentation.
    \end{theorem}

\begin{proof}
    \leanok
\end{proof}

\begin{theorem}\label{ext}
        \leanok
        \lean{ext}
        \uses{ValAlgHom,ValAlgEquiv}
                No documentation.
    \end{theorem}

\begin{proof}
    \leanok
\end{proof}

\begin{theorem}\label{bijective}
        \leanok
        \lean{bijective}
                No documentation.
    \end{theorem}

\begin{proof}
    \leanok
\end{proof}

\begin{theorem}\label{injective}
        \leanok
        \lean{injective}
                No documentation.
    \end{theorem}

\begin{proof}
    \leanok
\end{proof}

\begin{theorem}\label{surjective}
        \leanok
        \lean{surjective}
        \uses{ValRingEquiv,ValRingEquivClass,ValRingEquivClass.toValRingEquiv,ValAlgEquiv.refl,ValAlgEquiv.symm,ValAlgEquiv.trans,ValAlgEquiv.mk',ValAlgEquiv.coe_mk'}
                No documentation.
    \end{theorem}

\begin{proof}
    \leanok
\end{proof}

\begin{theorem}\label{map_le_map_iff_of_val_isEquiv_comap}
        \lean{map_le_map_iff_of_val_isEquiv_comap}
        \uses{Valued.le_iff_val_le,ValAlgEquiv.symm,ValAlgebra.val_isEquiv_comap}
                No documentation.
    \end{theorem}

\begin{theorem}\label{monotone_of_val_isEquiv_comap}
        \lean{monotone_of_val_isEquiv_comap}
        \uses{Valued.le_iff_val_le,ValAlgebra.val_isEquiv_comap}
                No documentation.
    \end{theorem}

\begin{theorem}\label{Valued.mem_nhds'}
        \lean{Valued.mem_nhds'}
                No documentation.
    \end{theorem}

\begin{theorem}\label{continuous_of_val_isEquiv_comap}
        \lean{continuous_of_val_isEquiv_comap}
        \uses{Valued.mem_nhds',ValAlgEquiv.symm,ValAlgebra.val_isEquiv_comap}
                No documentation.
    \end{theorem}

\begin{definition}\label{ValRingHom.mk'}
        \lean{ValRingHom.mk'}
        \uses{ValRingHom,monotone_of_val_isEquiv_comap,continuous_of_val_isEquiv_comap,ValAlgebra.val_isEquiv_comap,ValAlgEquiv.mk'}
                No documentation.
    \end{definition}

\begin{definition}\label{ValRingEquiv.mk'}
        \lean{ValRingEquiv.mk'}
        \uses{ValRingHom,ValRingEquiv,map_le_map_iff_of_val_isEquiv_comap,continuous_of_val_isEquiv_comap,ValAlgEquiv.symm,ValAlgebra.val_isEquiv_comap,ValAlgEquiv.mk'}
                No documentation.
    \end{definition}

\begin{definition}\label{ValRingHom.id}
        \leanok
        \lean{ValRingHom.id}
        \uses{ValRingHom,ValAlgEquiv.refl}
                The identity function as bundled valuation ring homomorphism.
    \end{definition}

\begin{theorem}\label{ValAlgHom.id_apply}
        \leanok
        \lean{ValAlgHom.id_apply}
        \uses{ValRingHom,ValRingHom.id,ValRingHom.comp,ValAlgHom}
                No documentation.
    \end{theorem}

\begin{proof}
    \leanok
\end{proof}

\begin{definition}\label{ValRingHom.comp}
        \leanok
        \lean{ValRingHom.comp}
        \uses{ValAlgEquiv.symm,ValAlgebra.val_isEquiv_comap}
                No documentation.
    \end{definition}

\begin{theorem}\label{ValAlgHom.id_comp}
        \leanok
        \lean{ValAlgHom.id_comp}
        \uses{ext,ValRingHom.id,ValRingHom.comp,ValAlgHom}
                No documentation.
    \end{theorem}

\begin{proof}
    \leanok
\end{proof}

\begin{theorem}\label{ValAlgHom.comp_id}
        \leanok
        \lean{ValAlgHom.comp_id}
        \uses{ext,ValRingHom.id,ValRingHom.comp,ValAlgHom}
                No documentation.
    \end{theorem}

\begin{proof}
    \leanok
\end{proof}

\begin{definition}\label{ValAlgEquiv.refl}
        \leanok
        \lean{ValAlgEquiv.refl}
        \uses{ValRingEquiv}
                Identity valued ring isomorphism.
    \end{definition}

\begin{theorem}\label{ValAlgEquiv.coe_refl}
        \leanok
        \lean{ValAlgEquiv.coe_refl}
        \uses{ValRingHom.id,ValAlgEquiv.refl}
                No documentation.
    \end{theorem}

\begin{proof}
    \leanok
\end{proof}

\begin{theorem}\label{ValRingEquiv.refl_apply}
        \leanok
        \lean{ValRingEquiv.refl_apply}
        \uses{ValAlgEquiv.refl}
                No documentation.
    \end{theorem}

\begin{proof}
    \leanok
\end{proof}

\begin{theorem}\label{ValRingEquiv.refl_toEquiv}
        \leanok
        \lean{ValRingEquiv.refl_toEquiv}
        \uses{ValAlgEquiv.refl}
                No documentation.
    \end{theorem}

\begin{proof}
    \leanok
\end{proof}

\begin{definition}\label{ValAlgEquiv.symm}
        \leanok
        \lean{ValAlgEquiv.symm}
        \uses{ValRingEquiv,ValAlgEquiv.symm_apply_apply}
                Inverse of a valued ring algebra equivalence.
    \end{definition}

\begin{theorem}\label{ValAlgEquiv.apply_symm_apply}
        \leanok
        \lean{ValAlgEquiv.apply_symm_apply}
        \uses{ValAlgEquiv.symm}
                No documentation.
    \end{theorem}

\begin{proof}
    \leanok
\end{proof}

\begin{theorem}\label{ValAlgEquiv.symm_apply_apply}
        \leanok
        \lean{ValAlgEquiv.symm_apply_apply}
        \uses{ValAlgEquiv.symm}
                No documentation.
    \end{theorem}

\begin{proof}
    \leanok
\end{proof}

\begin{theorem}\label{ValRingEquiv.symm_refl}
        \leanok
        \lean{ValRingEquiv.symm_refl}
        \uses{ValRingEquiv,ValAlgEquiv.refl,ValAlgEquiv.symm}
                No documentation.
    \end{theorem}

\begin{proof}
    \leanok
\end{proof}

\begin{theorem}\label{ValRingEquiv.apply_eq_iff_eq_symm_apply}
        \leanok
        \lean{ValRingEquiv.apply_eq_iff_eq_symm_apply}
        \uses{ValAlgEquiv.symm}
                No documentation.
    \end{theorem}

\begin{proof}
    \leanok
\end{proof}

\begin{theorem}\label{ValRingEquiv.symm_apply_eq}
        \leanok
        \lean{ValRingEquiv.symm_apply_eq}
        \uses{ValAlgEquiv.symm}
                No documentation.
    \end{theorem}

\begin{proof}
    \leanok
\end{proof}

\begin{theorem}\label{ValAlgEquiv.symm_symm}
        \leanok
        \lean{ValAlgEquiv.symm_symm}
        \uses{ext,ValAlgEquiv.symm}
                No documentation.
    \end{theorem}

\begin{proof}
    \leanok
\end{proof}

\begin{theorem}\label{ValAlgEquiv.symm_bijective}
        \leanok
        \lean{ValAlgEquiv.symm_bijective}
        \uses{bijective,ValAlgEquiv.symm,ValAlgEquiv.symm_symm}
                No documentation.
    \end{theorem}

\begin{proof}
    \leanok
\end{proof}

\begin{theorem}\label{ValRingEquiv.symm_injective}
        \leanok
        \lean{ValRingEquiv.symm_injective}
        \uses{bijective,injective,ValAlgEquiv.symm,ValAlgEquiv.symm_bijective}
                No documentation.
    \end{theorem}

\begin{proof}
    \leanok
\end{proof}

\begin{theorem}\label{ValRingEquiv.symm_surjective}
        \leanok
        \lean{ValRingEquiv.symm_surjective}
        \uses{bijective,surjective,ValAlgEquiv.symm,ValAlgEquiv.symm_bijective}
                No documentation.
    \end{theorem}

\begin{proof}
    \leanok
\end{proof}

\begin{theorem}\label{ValRingEquiv.toEquiv_symm}
        \leanok
        \lean{ValRingEquiv.toEquiv_symm}
        \uses{ValAlgEquiv.symm}
                No documentation.
    \end{theorem}

\begin{proof}
    \leanok
\end{proof}

\begin{definition}\label{ValAlgEquiv.trans}
        \leanok
        \lean{ValAlgEquiv.trans}
        \uses{ValRingEquiv}
                Algebra equivalences are transitive.
    \end{definition}

\begin{theorem}\label{ValAlgEquiv.coe_trans}
        \leanok
        \lean{ValAlgEquiv.coe_trans}
        \uses{ValAlgEquiv.trans}
                No documentation.
    \end{theorem}

\begin{proof}
    \leanok
\end{proof}

\begin{theorem}\label{ValAlgEquiv.trans_apply}
        \leanok
        \lean{ValAlgEquiv.trans_apply}
        \uses{ValAlgEquiv.trans}
                No documentation.
    \end{theorem}

\begin{proof}
    \leanok
\end{proof}

\begin{theorem}\label{ValRingEquiv.refl_trans}
        \leanok
        \lean{ValRingEquiv.refl_trans}
        \uses{ext,ValAlgEquiv.refl,ValAlgEquiv.trans}
                No documentation.
    \end{theorem}

\begin{proof}
    \leanok
\end{proof}

\begin{theorem}\label{ValRingEquiv.trans_refl}
        \leanok
        \lean{ValRingEquiv.trans_refl}
        \uses{ext,ValAlgEquiv.refl,ValAlgEquiv.trans}
                No documentation.
    \end{theorem}

\begin{proof}
    \leanok
\end{proof}

\begin{theorem}\label{ValAlgEquiv.symm_trans_apply}
        \leanok
        \lean{ValAlgEquiv.symm_trans_apply}
        \uses{ValAlgEquiv.symm,ValAlgEquiv.trans,ValAlgEquiv.trans_apply,ValRingEquiv.symm_trans}
                No documentation.
    \end{theorem}

\begin{proof}
    \leanok
\end{proof}

\begin{theorem}\label{ValRingEquiv.symm_trans}
        \leanok
        \lean{ValRingEquiv.symm_trans}
        \uses{ValAlgEquiv.symm,ValAlgEquiv.trans}
                No documentation.
    \end{theorem}

\begin{proof}
    \leanok
\end{proof}

\begin{theorem}\label{ValAlgEquiv.symm_trans_self}
        \leanok
        \lean{ValAlgEquiv.symm_trans_self}
        \uses{ext,ValAlgEquiv.refl,ValAlgEquiv.symm,ValAlgEquiv.symm_apply_apply,ValAlgEquiv.trans,ValRingEquiv.symm_trans}
                No documentation.
    \end{theorem}

\begin{proof}
    \leanok
\end{proof}

\begin{theorem}\label{ValAlgEquiv.self_symm_trans}
        \leanok
        \lean{ValAlgEquiv.self_symm_trans}
        \uses{ext,ValAlgEquiv.refl,ValAlgEquiv.symm,ValAlgEquiv.apply_symm_apply,ValAlgEquiv.trans,ValRingEquiv.symm_trans}
                No documentation.
    \end{theorem}

\begin{proof}
    \leanok
\end{proof}

\begin{theorem}\label{ValAlgEquiv.one_apply}
        \leanok
        \lean{ValAlgEquiv.one_apply}
                No documentation.
    \end{theorem}

\begin{proof}
    \leanok
\end{proof}

\begin{theorem}\label{ValAlgEquiv.mul_apply}
        \leanok
        \lean{ValAlgEquiv.mul_apply}
                No documentation.
    \end{theorem}

\begin{proof}
    \leanok
\end{proof}

\begin{definition}\label{valAlgebraMap}
        \leanok
        \lean{valAlgebraMap}
        \uses{ValRingHom,ValAlgebra}
                The valued ring homomorphism `R →+*v A` given by `Algebra` structure.
    \end{definition}

\begin{definition}\label{ValRingHom.toValAlgebra'}
        \leanok
        \lean{ValRingHom.toValAlgebra'}
        \uses{ValRingHom,ValAlgebra,ValRingHom.toValAlgebra}
                Creating a valued algebra from a valued ring morphism.
    \end{definition}

\begin{definition}\label{ValRingHom.toValAlgebra}
        \leanok
        \lean{ValRingHom.toValAlgebra}
        \uses{ValRingHom,ValAlgebra}
                No documentation.
    \end{definition}

\begin{theorem}\label{ValRingHom.valAlgebraMap_toValAlgebra}
        \leanok
        \lean{ValRingHom.valAlgebraMap_toValAlgebra}
        \uses{ValRingHom,ValAlgebra,valAlgebraMap}
                No documentation.
    \end{theorem}

\begin{proof}
    \leanok
\end{proof}

\begin{theorem}\label{ValAlgebra.valAlgebra_ext}
        \leanok
        \lean{ValAlgebra.valAlgebra_ext}
        \uses{ValAlgebra,ext,valAlgebraMap}
                To prove two valued algebra structures on a fixed `[CommRing R] [Ring A]` agree,
it suffices to check the `algebraMap`s agree.
    \end{theorem}

\begin{proof}
    \leanok
\end{proof}

\begin{theorem}\label{ValAlgebra.val_isEquiv_comap}
        \leanok
        \lean{ValAlgebra.val_isEquiv_comap}
        \uses{valAlgebraMap}
                No documentation.
    \end{theorem}

\begin{proof}
    \leanok
\end{proof}

\begin{theorem}\label{ValAlgebra.id.map_eq_id}
        \leanok
        \lean{ValAlgebra.id.map_eq_id}
        \uses{ValRingHom,ValRingHom.id,valAlgebraMap}
                The identity map inducing an `ValAlgebra` structure.
    \end{theorem}

\begin{proof}
    \leanok
\end{proof}

\begin{theorem}\label{ValAlgebra.id.map_eq_self}
        \leanok
        \lean{ValAlgebra.id.map_eq_self}
        \uses{ValAlgebra,ValRingHom.id,valAlgebraMap,ValAlgHom,ValAlgEquiv}
                No documentation.
    \end{theorem}

\begin{proof}
    \leanok
\end{proof}

\begin{definition}\label{ValAlgHom}
        \leanok
        \lean{ValAlgHom}
        \uses{ValRingHom,ext}
                Defining the homomorphism in the category of valued-`R`-algebra.
    \end{definition}

\begin{definition}\label{ValAlgEquiv}
        \leanok
        \lean{ValAlgEquiv}
        \uses{ValRingEquiv,ext}
                Defining the isomorphism in the category of valued-`R`-algebra.
    \end{definition}

\begin{definition}\label{ValAlgHomClass}
        \leanok
        \lean{ValAlgHomClass}
        \uses{ValRingHom,ValAlgebra,ValRingEquiv,ValRingHomClass,ValRingEquivClass,ext,injective,ValAlgebra.val_isEquiv_comap,ValAlgHom}
                `ValAlgHomClass F R A B` asserts `F` is a type of bundled valued algebra homomorphisms
from `A` to `B`.
    \end{definition}

\begin{definition}\label{ValAlgHom.ofAlgHomClassValRingHomClass}
        \leanok
        \lean{ValAlgHom.ofAlgHomClassValRingHomClass}
        \uses{ValRingHom,ValRingHomClass,ValRingHomClass.toValRingHom,ext,injective,ValAlgHom}
                No documentation.
    \end{definition}

\begin{theorem}\label{comp_valAlgebraMap}
        \leanok
        \lean{comp_valAlgebraMap}
        \uses{ValRingHom,ValRingHomClass,ext,ValRingHom.comp,valAlgebraMap,ValAlgEquiv}
                No documentation.
    \end{theorem}

\begin{proof}
    \leanok
\end{proof}

\begin{definition}\label{ValAlgEquiv.mk'}
        \lean{ValAlgEquiv.mk'}
        \uses{ValRingEquiv,ValRingEquiv.mk'}
                No documentation.
    \end{definition}

\begin{theorem}\label{ValAlgEquiv.coe_mk'}
        \lean{ValAlgEquiv.coe_mk'}
        \uses{ValAlgEquiv.refl,ValAlgEquiv.mk'}
                No documentation.
    \end{theorem}

\begin{theorem}\label{ValAlgHom.coe_id}
        \leanok
        \lean{ValAlgHom.coe_id}
        \uses{ValRingHom.id,ValAlgHom}
                Identity map as an `ValAlgHom`.
    \end{theorem}

\begin{proof}
    \leanok
\end{proof}

\begin{theorem}\label{ValAlgHom.id_toValRingHom}
        \leanok
        \lean{ValAlgHom.id_toValRingHom}
        \uses{ValRingHom,ValRingHom.id,ValAlgHom}
                No documentation.
    \end{theorem}

\begin{proof}
    \leanok
\end{proof}

\begin{theorem}\label{ValAlgHom.coe_comp}
        \leanok
        \lean{ValAlgHom.coe_comp}
        \uses{ValRingHom.comp}
                No documentation.
    \end{theorem}

\begin{proof}
    \leanok
\end{proof}

\begin{theorem}\label{ValAlgHom.comp_apply}
        \leanok
        \lean{ValAlgHom.comp_apply}
        \uses{ValRingHom.comp}
                No documentation.
    \end{theorem}

\begin{proof}
    \leanok
\end{proof}

\begin{theorem}\label{ValAlgHom.comp_toRingHom}
        \leanok
        \lean{ValAlgHom.comp_toRingHom}
        \uses{ValRingHom.comp}
                No documentation.
    \end{theorem}

\begin{proof}
    \leanok
\end{proof}

\begin{theorem}\label{ValAlgHom.comp_assoc}
        \leanok
        \lean{ValAlgHom.comp_assoc}
        \uses{ext,ValRingHom.id,ValRingHom.comp,ValAlgHom,ValAlgEquiv}
                No documentation.
    \end{theorem}

\begin{proof}
    \leanok
\end{proof}

\begin{theorem}\label{ValAlgEquiv.refl_toAlgHom}
        \leanok
        \lean{ValAlgEquiv.refl_toAlgHom}
        \uses{ValRingHom.id,ValAlgEquiv.refl}
                No documentation.
    \end{theorem}

\begin{proof}
    \leanok
\end{proof}

\begin{theorem}\label{ValAlgEquiv.symm_toEquiv_eq_symm}
        \leanok
        \lean{ValAlgEquiv.symm_toEquiv_eq_symm}
        \uses{ValAlgEquiv.symm}
                No documentation.
    \end{theorem}

\begin{proof}
    \leanok
\end{proof}

\begin{theorem}\label{ValAlgEquiv.invFun_eq_symm}
        \leanok
        \lean{ValAlgEquiv.invFun_eq_symm}
        \uses{ValAlgEquiv.symm}
                No documentation.
    \end{theorem}

\begin{proof}
    \leanok
\end{proof}

\begin{theorem}\label{ValAlgEquiv.refl_symm}
        \leanok
        \lean{ValAlgEquiv.refl_symm}
        \uses{ValAlgEquiv.refl,ValAlgEquiv.symm}
                No documentation.
    \end{theorem}

\begin{proof}
    \leanok
\end{proof}

\begin{theorem}\label{ValAlgEquiv.toRingEquiv_symm}
        \leanok
        \lean{ValAlgEquiv.toRingEquiv_symm}
        \uses{ValAlgEquiv.symm}
                No documentation.
    \end{theorem}

\begin{proof}
    \leanok
\end{proof}

\begin{theorem}\label{ValAlgEquiv.symm_toRingEquiv}
        \leanok
        \lean{ValAlgEquiv.symm_toRingEquiv}
        \uses{ValAlgEquiv.symm}
                No documentation.
    \end{theorem}

\begin{proof}
    \leanok
\end{proof}

\begin{theorem}\label{ValAlgEquiv.comp_symm}
        \leanok
        \lean{ValAlgEquiv.comp_symm}
        \uses{ext,ValRingHom.id,ValAlgHom.id_apply,ValRingHom.comp,ValAlgEquiv.symm,ValAlgEquiv.apply_symm_apply,ValAlgHom,ValAlgHom.coe_comp,ValAlgHom.comp_apply}
                No documentation.
    \end{theorem}

\begin{proof}
    \leanok
\end{proof}

\begin{theorem}\label{ValAlgEquiv.symm_comp}
        \leanok
        \lean{ValAlgEquiv.symm_comp}
        \uses{ext,ValRingHom.id,ValAlgHom.id_apply,ValRingHom.comp,ValAlgEquiv.symm,ValAlgEquiv.symm_apply_apply,ValAlgHom,ValAlgHom.coe_comp,ValAlgHom.comp_apply}
                No documentation.
    \end{theorem}

\begin{proof}
    \leanok
\end{proof}

\begin{theorem}\label{ValAlgEquiv.leftInverse_symm}
        \leanok
        \lean{ValAlgEquiv.leftInverse_symm}
        \uses{ValAlgEquiv.symm}
                No documentation.
    \end{theorem}

\begin{proof}
    \leanok
\end{proof}

\begin{theorem}\label{ValAlgEquiv.rightInverse_symm}
        \leanok
        \lean{ValAlgEquiv.rightInverse_symm}
        \uses{ValAlgebra,ValRingEquiv,bijective,ValRingHom.comp,ValAlgEquiv.refl,ValAlgEquiv.symm,ValAlgEquiv.trans,ValRingEquiv.symm_trans,ValAlgEquiv.symm_trans_self}
                No documentation.
    \end{theorem}

\begin{proof}
    \leanok
\end{proof}

\begin{theorem}\label{ValAlgEquiv.one_def}
        \leanok
        \lean{ValAlgEquiv.one_def}
        \uses{ValAlgEquiv.refl}
                No documentation.
    \end{theorem}

\begin{proof}
    \leanok
\end{proof}

\begin{definition}\label{ValAlgEquiv.toAlgEquivₘ}
        \leanok
        \lean{ValAlgEquiv.toAlgEquivₘ}
        \uses{ext,ValAlgEquiv}
                No documentation.
    \end{definition}

\begin{theorem}\label{ValAlgEquiv.toAlgEquivₘ_coe}
        \leanok
        \lean{ValAlgEquiv.toAlgEquivₘ_coe}
        \uses{ext,ValAlgEquiv.toAlgEquivₘ}
                No documentation.
    \end{theorem}

\begin{proof}
    \leanok
\end{proof}

\begin{theorem}\label{ValAlgEquiv.toAlgEquiv_injective}
        \leanok
        \lean{ValAlgEquiv.toAlgEquiv_injective}
        \uses{injective,ValAlgHom,ValAlgEquiv,ValAlgEquiv.toAlgEquivₘ}
                No documentation.
    \end{theorem}

\begin{proof}
    \leanok
\end{proof}

\begin{definition}\label{liftInteger}
        \leanok
        \lean{liftInteger}
        \uses{ValAlgebra,ValRingEquiv,ext}
                No documentation.
    \end{definition}

\begin{definition}\label{liftValuationSubring}
        \leanok
        \lean{liftValuationSubring}
        \uses{liftInteger}
                No documentation.
    \end{definition}

\begin{definition}\label{liftResidueField}
        \lean{liftResidueField}
        \uses{Valued.ResidueField,ValRingEquiv,ValRingHom.id,ValAlgHom,ValAlgEquiv,liftInteger}
                No documentation.
    \end{definition}

\begin{definition}\label{ValAlgHom.liftValuationIntegerQuotientleIdeal}
        \lean{ValAlgHom.liftValuationIntegerQuotientleIdeal}
        \uses{AlgHom.liftValuationInteger,ValAlgHom}
                No documentation.
    \end{definition}

\begin{definition}\label{ValAlgIso.liftValuationIntegerQuotientleIdeal}
        \lean{ValAlgIso.liftValuationIntegerQuotientleIdeal}
                No documentation.
    \end{definition}

\begin{theorem}\label{ValAlgEquiv.coe_liftInteger}
        \leanok
        \lean{ValAlgEquiv.coe_liftInteger}
        \uses{liftInteger}
                No documentation.
    \end{theorem}

\begin{proof}
    \leanok
\end{proof}

\begin{theorem}\label{ValAlgEquiv.liftInteger_refl}
        \leanok
        \lean{ValAlgEquiv.liftInteger_refl}
        \uses{ValAlgebra,ext,ValAlgEquiv.refl,liftInteger}
                No documentation.
    \end{theorem}

\begin{proof}
    \leanok
\end{proof}

\begin{theorem}\label{ValAlgEquiv.eq_refl_of_liftInteger_eq_refl}
        \leanok
        \lean{ValAlgEquiv.eq_refl_of_liftInteger_eq_refl}
        \uses{ext,surjective,ValAlgEquiv.refl,ValAlgEquiv,liftInteger}
                No documentation.
    \end{theorem}

\begin{proof}
    \leanok
\end{proof}

\begin{theorem}\label{unique_of_valuation_extension}
        \lean{unique_of_valuation_extension}
        \uses{ValAlgebra,ext}
                No documentation.
    \end{theorem}

\begin{definition}\label{toValAlgEquiv}
        \lean{toValAlgEquiv}
        \uses{ValAlgEquiv}
                No documentation.
    \end{definition}

\begin{theorem}\label{LocalRing.algebraMap_residue_compat}
        \leanok
        \lean{LocalRing.algebraMap_residue_compat}
        \uses{Valued.ResidueField,ValRingHom.id,ValRingHom.comp}
                No documentation.
    \end{theorem}

\begin{proof}
    \leanok
\end{proof}

\begin{theorem}\label{LocalRing.residue_irreducible_eq_zero}
        \leanok
        \lean{LocalRing.residue_irreducible_eq_zero}
        \uses{Valued.maximalIdeal,ValRingHom.id}
                No documentation.
    \end{theorem}

\begin{proof}
    \leanok
\end{proof}

\begin{theorem}\label{LocalRing.is_unit_iff_residue_ne_zero}
        \leanok
        \lean{LocalRing.is_unit_iff_residue_ne_zero}
        \uses{ValRingHom.id}
                No documentation.
    \end{theorem}

\begin{proof}
    \leanok
\end{proof}

\begin{theorem}\label{LocalRing.residue_eq_add_irreducible}
        \leanok
        \lean{LocalRing.residue_eq_add_irreducible}
        \uses{ValRingHom.id,LocalRing.residue_irreducible_eq_zero}
                No documentation.
    \end{theorem}

\begin{proof}
    \leanok
\end{proof}

\begin{theorem}\label{LocalRing.is_unit_of_unit_add_nonunit}
        \leanok
        \lean{LocalRing.is_unit_of_unit_add_nonunit}
                No documentation.
    \end{theorem}

\begin{proof}
    \leanok
\end{proof}

\begin{theorem}\label{LocalRing.maximalIdeal_eq_jacobson_of_bot}
        \lean{LocalRing.maximalIdeal_eq_jacobson_of_bot}
        \uses{Valued.maximalIdeal,maximalIdeal_eq,ValAlgEquiv.symm}
                No documentation.
    \end{theorem}

\begin{theorem}\label{DiscreteValuationRing.unit_mul_irreducible_of_irreducible}
        \leanok
        \lean{DiscreteValuationRing.unit_mul_irreducible_of_irreducible}
                No documentation.
    \end{theorem}

\begin{proof}
    \leanok
\end{proof}

\begin{theorem}\label{DiscreteValuationRing.mul_irreducible_of_not_unit}
        \leanok
        \lean{DiscreteValuationRing.mul_irreducible_of_not_unit}
        \uses{Valued.maximalIdeal,ValAlgEquiv.symm}
                No documentation.
    \end{theorem}

\begin{proof}
    \leanok
\end{proof}

\begin{theorem}\label{DiscreteValuationRing.mul_irreducible_square_of_not_unit_of_not_irreducible}
        \leanok
        \lean{DiscreteValuationRing.mul_irreducible_square_of_not_unit_of_not_irreducible}
        \uses{ValRingHom.id,DiscreteValuationRing.mul_irreducible_of_not_unit}
                No documentation.
    \end{theorem}

\begin{proof}
    \leanok
\end{proof}

\begin{theorem}\label{DiscreteValuationRing.irreducible_of_irreducible_add_addVal_ge_two}
        \leanok
        \lean{DiscreteValuationRing.irreducible_of_irreducible_add_addVal_ge_two}
        \uses{LocalRing.is_unit_of_unit_add_nonunit,DiscreteValuationRing.unit_mul_irreducible_of_irreducible}
                No documentation.
    \end{theorem}

\begin{proof}
    \leanok
\end{proof}

\begin{theorem}\label{DiscreteValuationRing.maximalIdeal_pow_eq_span_irreducible_pow}
        \lean{DiscreteValuationRing.maximalIdeal_pow_eq_span_irreducible_pow}
        \uses{Valued.maximalIdeal,maximalIdeal_eq}
                No documentation.
    \end{theorem}

\begin{definition}\label{AlgEquiv.restrictScalarsHom}
        \leanok
        \lean{AlgEquiv.restrictScalarsHom}
        \uses{ext}
                No documentation.
    \end{definition}

\begin{theorem}\label{PowerBasis.algEquiv_ext}
        \leanok
        \lean{PowerBasis.algEquiv_ext}
        \uses{ext}
                No documentation.
    \end{theorem}

\begin{proof}
    \leanok
\end{proof}

\begin{definition}\label{Ideal.quotientAlgebra'}
        \leanok
        \lean{Ideal.quotientAlgebra'}
                No documentation.
    \end{definition}

\begin{theorem}\label{Int.eq_neg_one_of_dvd_neg_one}
        \leanok
        \lean{Int.eq_neg_one_of_dvd_neg_one}
                No documentation.
    \end{theorem}

\begin{proof}
    \leanok
\end{proof}

\begin{theorem}\label{WithZero.ofAdd_eq_neg_one_of_pow_eq_neg_one}
        \lean{WithZero.ofAdd_eq_neg_one_of_pow_eq_neg_one}
        \uses{Valued.zero_le,Int.eq_neg_one_of_dvd_neg_one}
                No documentation.
    \end{theorem}

\begin{lemma}\label{eq_one_of_eq_one_of_le_one_le_one}
        \leanok
        \lean{eq_one_of_eq_one_of_le_one_le_one}
        \uses{ValAlgEquiv.symm}
                No documentation.
    \end{lemma}

\begin{proof}
    \leanok
\end{proof}

\begin{theorem}\label{map_zpow}
        \leanok
        \lean{map_zpow}
        \uses{Valued.valuationSubring}
                No documentation.
    \end{theorem}

\begin{proof}
    \leanok
\end{proof}

\begin{theorem}\label{val_valuationSubring_unit}
        \leanok
        \lean{val_valuationSubring_unit}
        \uses{Valued.valuationSubring}
                No documentation.
    \end{theorem}

\begin{proof}
    \leanok
\end{proof}

\begin{theorem}\label{isUnit_in_valuationSubring_of_val_eq_one}
        \leanok
        \lean{isUnit_in_valuationSubring_of_val_eq_one}
        \uses{Valued.valuationSubring}
                No documentation.
    \end{theorem}

\begin{proof}
    \leanok
\end{proof}

\begin{theorem}\label{unitOfValOne_elem}
        \leanok
        \lean{unitOfValOne_elem}
        \uses{Valued.valuationSubring,isUnit_in_valuationSubring_of_val_eq_one}
                create a term of `v.valuationSubringˣ` from a term `x : K` with `v x = 1`
    \end{theorem}

\begin{proof}
    \leanok
\end{proof}

\begin{theorem}\label{val_unitOfValOne_eq_one}
        \leanok
        \lean{val_unitOfValOne_eq_one}
        \uses{unitOfValOne_elem}
                No documentation.
    \end{theorem}

\begin{proof}
    \leanok
\end{proof}

\begin{theorem}\label{pow_Uniformizer_all}
        \leanok
        \lean{pow_Uniformizer_all}
        \uses{Valued.valuationSubring,ext,ValRingHom.id,ValAlgEquiv.symm}
                No documentation.
    \end{theorem}

\begin{proof}
    \leanok
\end{proof}

\begin{theorem}\label{pow_Uniformizer'}
        \leanok
        \lean{pow_Uniformizer'}
        \uses{Valued.valuationSubring,ext,ValRingHom.id}
                No documentation.
    \end{theorem}

\begin{proof}
    \leanok
\end{proof}

\begin{theorem}\label{val_pow_Uniformizer}
        \leanok
        \lean{val_pow_Uniformizer}
        \uses{Valued.valuationSubring,val_valuationSubring_unit}
                No documentation.
    \end{theorem}

\begin{proof}
    \leanok
\end{proof}

\begin{theorem}\label{val_pow_Uniformizer_all}
        \leanok
        \lean{val_pow_Uniformizer_all}
        \uses{Valued.valuationSubring,map_zpow,val_valuationSubring_unit,pow_Uniformizer_all,val_pow_Uniformizer}
                No documentation.
    \end{theorem}

\begin{proof}
    \leanok
\end{proof}

\begin{theorem}\label{val_pow_Uniformizer_all'}
        \lean{val_pow_Uniformizer_all'}
        \uses{Valued.valuationSubring,ValAlgEquiv.mk',pow_Uniformizer_all,val_pow_Uniformizer,val_pow_Uniformizer_all}
                No documentation.
    \end{theorem}

\begin{lemma}\label{isUniformizer_of_uniformizer_of_le_one_le_one}
        \lean{isUniformizer_of_uniformizer_of_le_one_le_one}
        \uses{Valued.valuationSubring,ValAlgEquiv.symm,WithZero.ofAdd_eq_neg_one_of_pow_eq_neg_one,eq_one_of_eq_one_of_le_one_le_one,map_zpow,val_valuationSubring_unit,pow_Uniformizer_all}
                If `π : K` is a uniformizer for `v`, and `v x ≤ 1 → v' x ≤ 1, ∀ x : K`, then `π` is also a uniformizer for `v'`.
    \end{lemma}

\begin{theorem}\label{isUniformizer_of_uniformizer_of_equiv}
        \lean{isUniformizer_of_uniformizer_of_equiv}
        \uses{isUniformizer_of_uniformizer_of_le_one_le_one}
                If `π : K` is a uniformizer for `v`, and `v` is equivalent to `v'`, then `π` is also a uniformizer for `v'`.
    \end{theorem}

\begin{theorem}\label{val_pow_Uniformizer_all_of_equiv}
        \lean{val_pow_Uniformizer_all_of_equiv}
        \uses{Valued.valuationSubring,map_zpow,val_valuationSubring_unit,pow_Uniformizer_all,val_pow_Uniformizer,val_pow_Uniformizer_all,isUniformizer_of_uniformizer_of_equiv}
                No documentation.
    \end{theorem}

\begin{theorem}\label{lt_one_lt_one_of_le_one_le_one}
        \lean{lt_one_lt_one_of_le_one_le_one}
        \uses{Valued.valuationSubring,ValAlgEquiv.mk',eq_one_of_eq_one_of_le_one_le_one,val_valuationSubring_unit,unitOfValOne_elem,pow_Uniformizer',val_pow_Uniformizer,isUniformizer_of_uniformizer_of_le_one_le_one}
                No documentation.
    \end{theorem}

\begin{theorem}\label{isEquiv_of_le_one_le_one}
        \lean{isEquiv_of_le_one_le_one}
        \uses{Valued.zero_le,lt_one_lt_one_of_le_one_le_one}
                Two discrete valuations `v` and `v'` on a field `K` are equivalent, if `v x ≤ 1 → v' x ≤ 1, ∀x : K`.
    \end{theorem}

\begin{theorem}\label{isEquiv_iff_eq}
        \lean{isEquiv_iff_eq}
        \uses{ext,pow_Uniformizer_all,val_pow_Uniformizer,val_pow_Uniformizer_all,val_pow_Uniformizer_all_of_equiv}
                For discrete valuations, being equivalent is the same as being equal.
    \end{theorem}

\begin{definition}\label{Valuation.Nontrivial}
        \leanok
        \lean{Valuation.Nontrivial}
                No documentation.
    \end{definition}

\begin{theorem}\label{Valuation.nontrivial_def}
        \leanok
        \lean{Valuation.nontrivial_def}
        \uses{Valuation.Nontrivial}
                No documentation.
    \end{theorem}

\begin{proof}
    \leanok
\end{proof}

\begin{theorem}\label{DiscreteValuation.valuationSubring_DVR_of_equiv_discrete}
        \leanok
        \lean{DiscreteValuation.valuationSubring_DVR_of_equiv_discrete}
        \uses{Valued.valuationSubring}
                No documentation.
    \end{theorem}

\begin{proof}
    \leanok
\end{proof}

\begin{definition}\label{DiscreteValuation.ofNontrivial}
        \lean{DiscreteValuation.ofNontrivial}
        \uses{Valuation.Nontrivial}
                No documentation.
    \end{definition}

\begin{theorem}\label{DiscreteValuation.isEquiv_ofNontrivial}
        \lean{DiscreteValuation.isEquiv_ofNontrivial}
        \uses{Valued.valuationSubring,Valued.maximalIdeal,ValRingHom.comp,Valuation.Nontrivial,DiscreteValuation.valuationSubring_DVR_of_equiv_discrete,DiscreteValuation.ofNontrivial}
                No documentation.
    \end{theorem}

\begin{theorem}\label{Algebra.pow_mem_adjoin_singleton}
        \leanok
        \lean{Algebra.pow_mem_adjoin_singleton}
                No documentation.
    \end{theorem}

\begin{proof}
    \leanok
\end{proof}

\begin{theorem}\label{ExtDVR.exists_lift_residue_primitive}
        \leanok
        \lean{ExtDVR.exists_lift_residue_primitive}
        \uses{Valued.ResidueField,surjective,ValRingHom.id,ExtDVR.exists_primitive}
                There exists `x : B` generating `k_B` over `k_A`
    \end{theorem}

\begin{proof}
    \leanok
\end{proof}

\begin{theorem}\label{ExtDVR.exists_lift_polynomial_of_residue}
        \leanok
        \lean{ExtDVR.exists_lift_polynomial_of_residue}
        \uses{Valued.ResidueField,surjective,ValRingHom.id}
                For any `x : k_B`, there eixsts `f : A[X]` that reduces to the minimal polynomial of `x` over `k_A`
    \end{theorem}

\begin{proof}
    \leanok
\end{proof}

\begin{lemma}\label{ExtDVR.adjoin_lift_residue_primitive_and_irreducible_sup_maximalIdeal_eq_top}
        \lean{ExtDVR.adjoin_lift_residue_primitive_and_irreducible_sup_maximalIdeal_eq_top}
        \uses{eq_top,Valued.maximalIdeal,maximalIdeal_eq,Valued.ResidueField,surjective,ValRingHom.id,ValRingHom.comp,LocalRing.algebraMap_residue_compat}
                Auxiliary lemma: `A[x, ϖ] ⊔ m_B = ⊤`. Can be strenthened to `A[x] ⊔ m_B = B`
    \end{lemma}

\begin{lemma}\label{ExtDVR.adjoin_lift_residue_primitive_and_irreducible_sup_maximalIdeal_pow_eq_top}
        \lean{ExtDVR.adjoin_lift_residue_primitive_and_irreducible_sup_maximalIdeal_pow_eq_top}
        \uses{eq_top,Valued.maximalIdeal,maximalIdeal_eq,ValRingHom.id,DiscreteValuationRing.maximalIdeal_pow_eq_span_irreducible_pow,Algebra.pow_mem_adjoin_singleton,ExtDVR.adjoin_lift_residue_primitive_and_irreducible_sup_maximalIdeal_eq_top}
                Auxiliary lemma: `A[x, ϖ] ⊔ m_B ^ i = ⊤` for any `i : ℕ`
    \end{lemma}

\begin{theorem}\label{ExtDVR.maximalIdeal_map_ne_bot_of_injective}
        \leanok
        \lean{ExtDVR.maximalIdeal_map_ne_bot_of_injective}
        \uses{Valued.maximalIdeal,injective}
                `m_A • B ≠ ⊥`
    \end{theorem}

\begin{proof}
    \leanok
\end{proof}

\begin{lemma}\label{ExtDVR.3}
        \lean{ExtDVR.3}
        \uses{UpperRamificationGroup.basis}
                No documentation.
    \end{lemma}

\begin{theorem}\label{ExtDVR.adjoin_lift_residue_primitive_and_irreducible_eq_top}
        \lean{ExtDVR.adjoin_lift_residue_primitive_and_irreducible_eq_top}
        \uses{eq_top,Valued.maximalIdeal,injective,ValRingHom.id,ValAlgEquiv.symm,ExtDVR.adjoin_lift_residue_primitive_and_irreducible_sup_maximalIdeal_pow_eq_top,ExtDVR.maximalIdeal_map_ne_bot_of_injective}
                No documentation.
    \end{theorem}

\begin{theorem}\label{ExtDVR.residue_primitive_of_add_uniformizer}
        \leanok
        \lean{ExtDVR.residue_primitive_of_add_uniformizer}
        \uses{Valued.ResidueField,ValRingHom.id,LocalRing.residue_eq_add_irreducible}
                No documentation.
    \end{theorem}

\begin{proof}
    \leanok
\end{proof}

\begin{theorem}\label{ExtDVR.not_unit_aeval_lift_residue_primitive}
        \leanok
        \lean{ExtDVR.not_unit_aeval_lift_residue_primitive}
        \uses{ValRingHom.id,ValRingHom.comp,LocalRing.algebraMap_residue_compat,LocalRing.is_unit_iff_residue_ne_zero}
                No documentation.
    \end{theorem}

\begin{proof}
    \leanok
\end{proof}

\begin{theorem}\label{ExtDVR.irreducible_aeval_lift_redisue_primitive_add_irreducible_of_reducible_aeval_lift_residue_primitve}
        \leanok
        \lean{ExtDVR.irreducible_aeval_lift_redisue_primitive_add_irreducible_of_reducible_aeval_lift_residue_primitve}
        \uses{ValRingHom.id,ValRingHom.comp,ValAlgEquiv.one_apply,LocalRing.algebraMap_residue_compat,LocalRing.is_unit_iff_residue_ne_zero,DiscreteValuationRing.mul_irreducible_square_of_not_unit_of_not_irreducible,DiscreteValuationRing.irreducible_of_irreducible_add_addVal_ge_two,ExtDVR.not_unit_aeval_lift_residue_primitive,Polynomial.taylor_order_one,Polynomial.taylor_order_one_apply,Polynomial.taylor_order_one_apply₂}
                This is the first part of lemma 4:
If `f x` has valuation ≥ 2, then `f (x + ϖ)` is a uniformizer.
    \end{theorem}

\begin{proof}
    \leanok
\end{proof}

\begin{theorem}\label{ExtDVR.adjoin_lift_primitive_eq_top_of_irreducible_aeval_lift_residue_primitive}
        \lean{ExtDVR.adjoin_lift_primitive_eq_top_of_irreducible_aeval_lift_residue_primitive}
        \uses{eq_top,Valued.ResidueField,ValRingHom.id,ValAlgEquiv.refl,ExtDVR.adjoin_lift_residue_primitive_and_irreducible_eq_top}
                This is the second part of lemma 4:
`B = A[x]` if `k_B = k_A[x]` and `f x` is a uniformizer.
    \end{theorem}

\begin{theorem}\label{ExtDVR.exists_primitive}
        \lean{ExtDVR.exists_primitive}
        \uses{eq_top,ValRingHom.id,ExtDVR.exists_lift_residue_primitive,ExtDVR.exists_lift_polynomial_of_residue,ExtDVR.residue_primitive_of_add_uniformizer,ExtDVR.irreducible_aeval_lift_redisue_primitive_add_irreducible_of_reducible_aeval_lift_residue_primitve,ExtDVR.adjoin_lift_primitive_eq_top_of_irreducible_aeval_lift_residue_primitive}
                For a finite extension of DVR `A ↪ B` with seperable residue field extension,
there exists `x : B` s.t. `B = A[x]`
    \end{theorem}

\begin{theorem}\label{Polynomial.hasseDeriv_coeff_zero}
        \leanok
        \lean{Polynomial.hasseDeriv_coeff_zero}
                No documentation.
    \end{theorem}

\begin{proof}
    \leanok
\end{proof}

\begin{theorem}\label{Polynomial.hasseDeriv_coeff_one}
        \leanok
        \lean{Polynomial.hasseDeriv_coeff_one}
                No documentation.
    \end{theorem}

\begin{proof}
    \leanok
\end{proof}

\begin{theorem}\label{Polynomial.X_sq_dvd_hassDeriv_sub_coeff_sub_derivative_coeff}
        \leanok
        \lean{Polynomial.X_sq_dvd_hassDeriv_sub_coeff_sub_derivative_coeff}
        \uses{Polynomial.hasseDeriv_coeff_zero,Polynomial.hasseDeriv_coeff_one}
                No documentation.
    \end{theorem}

\begin{proof}
    \leanok
\end{proof}

\begin{theorem}\label{Polynomial.taylor_order_one}
        \leanok
        \lean{Polynomial.taylor_order_one}
        \uses{ValRingHom.comp,Polynomial.X_sq_dvd_hassDeriv_sub_coeff_sub_derivative_coeff}
                No documentation.
    \end{theorem}

\begin{proof}
    \leanok
\end{proof}

\begin{theorem}\label{Polynomial.taylor_order_one_apply}
        \leanok
        \lean{Polynomial.taylor_order_one_apply}
        \uses{ValRingHom.comp,ValAlgEquiv.one_apply,Polynomial.taylor_order_one}
                No documentation.
    \end{theorem}

\begin{proof}
    \leanok
\end{proof}

\begin{theorem}\label{Polynomial.taylor_order_one_apply₂}
        \leanok
        \lean{Polynomial.taylor_order_one_apply₂}
        \uses{ValAlgEquiv.one_apply,Polynomial.taylor_order_one,Polynomial.taylor_order_one_apply}
                No documentation.
    \end{theorem}

\begin{proof}
    \leanok
\end{proof}

\begin{theorem}\label{Polynomial.taylor_order_one_apply_aeval}
        \leanok
        \lean{Polynomial.taylor_order_one_apply_aeval}
        \uses{ValAlgEquiv.one_apply,Polynomial.taylor_order_one,Polynomial.taylor_order_one_apply,Polynomial.taylor_order_one_apply₂}
                No documentation.
    \end{theorem}

\begin{proof}
    \leanok
\end{proof}

\begin{theorem}\label{Polynomial.taylor_order_zero_apply_aeval}
        \leanok
        \lean{Polynomial.taylor_order_zero_apply_aeval}
        \uses{ValAlgEquiv.one_apply,Polynomial.taylor_order_one,Polynomial.taylor_order_one_apply,Polynomial.taylor_order_one_apply_aeval}
                No documentation.
    \end{theorem}

\begin{proof}
    \leanok
\end{proof}

\begin{theorem}\label{integer_valuation_eq}
        \leanok
        \lean{integer_valuation_eq}
                No documentation.
    \end{theorem}

\begin{proof}
    \leanok
\end{proof}

\begin{theorem}\label{integerAlgebraMap.monotone}
        \lean{integerAlgebraMap.monotone}
                No documentation.
    \end{theorem}

\begin{theorem}\label{integer_val_coe}
        \leanok
        \lean{integer_val_coe}
                No documentation.
    \end{theorem}

\begin{proof}
    \leanok
\end{proof}

\begin{theorem}\label{valuationSubring_val_coe}
        \leanok
        \lean{valuationSubring_val_coe}
        \uses{Valued.valuationSubring}
                No documentation.
    \end{theorem}

\begin{proof}
    \leanok
\end{proof}

\begin{theorem}\label{integer_val_le_one}
        \leanok
        \lean{integer_val_le_one}
                No documentation.
    \end{theorem}

\begin{proof}
    \leanok
\end{proof}

\begin{theorem}\label{maximalIdeal_eq}
        \lean{maximalIdeal_eq}
        \uses{Valued.maximalIdeal,ValRingHom.id}
                No documentation.
    \end{theorem}

\begin{theorem}\label{Valuation.isEquiv_iff_integer}
        \leanok
        \lean{Valuation.isEquiv_iff_integer}
        \uses{Valued.valuationSubring,ext}
                No documentation.
    \end{theorem}

\begin{proof}
    \leanok
\end{proof}

\begin{theorem}\label{Valuation.eval_lt_one_of_coeff_le_one_of_const_eq_zero_of_lt_one}
        \leanok
        \lean{Valuation.eval_lt_one_of_coeff_le_one_of_const_eq_zero_of_lt_one}
                No documentation.
    \end{theorem}

\begin{proof}
    \leanok
\end{proof}

\begin{theorem}\label{Valuation.aeval_valuationSubring_lt_one_of_lt_one
}
        \leanok
        \lean{Valuation.aeval_valuationSubring_lt_one_of_lt_one
}
        \uses{Valued.valuationSubring,Valuation.eval_lt_one_of_coeff_le_one_of_const_eq_zero_of_lt_one}
                No documentation.
    \end{theorem}

\begin{proof}
    \leanok
\end{proof}

\begin{theorem}\label{Valuation.aeval_valuationSubring_lt_one_of_lt_one_self
}
        \leanok
        \lean{Valuation.aeval_valuationSubring_lt_one_of_lt_one_self
}
        \uses{Valued.valuationSubring,Valuation.eval_lt_one_of_coeff_le_one_of_const_eq_zero_of_lt_one}
                No documentation.
    \end{theorem}

\begin{proof}
    \leanok
\end{proof}

\begin{theorem}\label{Valuation.mem_integer_of_mem_integral_closure}
        \leanok
        \lean{Valuation.mem_integer_of_mem_integral_closure}
        \uses{Valued.valuationSubring,ValAlgEquiv.trans}
                No documentation.
    \end{theorem}

\begin{proof}
    \leanok
\end{proof}

\begin{theorem}\label{nontrivial_of_valuation_extension}
        \leanok
        \lean{nontrivial_of_valuation_extension}
        \uses{ext,ValRingHom.id,Valuation.Nontrivial}
                No documentation.
    \end{theorem}

\begin{proof}
    \leanok
\end{proof}

\begin{theorem}\label{extension_valuation_equiv_extendedValuation_of_discrete
}
        \lean{extension_valuation_equiv_extendedValuation_of_discrete
}
        \uses{Valued.valuationSubring,ext,ValAlgEquiv.symm,ValAlgEquiv.trans,isEquiv_of_le_one_le_one,Valuation.Nontrivial,DiscreteValuation.ofNontrivial,DiscreteValuation.isEquiv_ofNontrivial,Valuation.mem_integer_of_mem_integral_closure,nontrivial_of_valuation_extension}
                If a valuation `v : L → ℤₘ₀` extends a discrete valuation on `K`, then `v` is equivalent to `extendedValuation K L`.
    \end{theorem}

\begin{theorem}\label{extension_integer_eq_extendedValuation_of_discrete}
        \leanok
        \lean{extension_integer_eq_extendedValuation_of_discrete}
        \uses{ext,Valuation.isEquiv_iff_integer}
                No documentation.
    \end{theorem}

\begin{proof}
    \leanok
\end{proof}

\begin{theorem}\label{integral_closure_eq_integer_of_complete_discrete
}
        \leanok
        \lean{integral_closure_eq_integer_of_complete_discrete
}
        \uses{Valued.valuationSubring,ext,ValRingHom.comp,extension_integer_eq_extendedValuation_of_discrete}
                No documentation.
    \end{theorem}

\begin{proof}
    \leanok
\end{proof}

\begin{theorem}\label{unique_valuationSubring_of_ext}
        \leanok
        \lean{unique_valuationSubring_of_ext}
        \uses{Valued.valuationSubring,ext,ValRingHom.comp}
                No documentation.
    \end{theorem}

\begin{proof}
    \leanok
\end{proof}

\begin{theorem}\label{unique_val_of_ext
}
        \leanok
        \lean{unique_val_of_ext
}
        \uses{unique_valuationSubring_of_ext,Valuation.unique_val_of_ext,Valued.valuationSubring,ext,IsValExtension}
                No documentation.
    \end{theorem}

\begin{proof}
    \leanok
\end{proof}

\begin{theorem}\label{DiscreteValuation.algHom_preserve_val_of_complete}
        \leanok
        \lean{DiscreteValuation.algHom_preserve_val_of_complete}
        \uses{Valuation.unique_val_of_ext,ext,ValRingHom.comp,ValAlgebra.val_isEquiv_comap,IsValExtension}
                No documentation.
    \end{theorem}

\begin{proof}
    \leanok
\end{proof}

\begin{theorem}\label{DiscreteValuation.algEquiv_preserve_val_of_complete}
        \leanok
        \lean{DiscreteValuation.algEquiv_preserve_val_of_complete}
        \uses{ValAlgEquiv.algEquiv_preserve_val,ValRingHom.comp,DiscreteValuation.algHom_preserve_val_of_complete}
                No documentation.
    \end{theorem}

\begin{proof}
    \leanok
\end{proof}

\begin{definition}\label{DiscreteValuation.fromAlgEquiv}
        \lean{DiscreteValuation.fromAlgEquiv}
        \uses{ValAlgEquiv.algEquiv_preserve_val,ValAlgEquiv.mk'}
                No documentation.
    \end{definition}

\begin{definition}\label{DiscreteValuation.equivAlgEquiv}
        \lean{DiscreteValuation.equivAlgEquiv}
        \uses{DiscreteValuation.fromAlgEquiv}
                No documentation.
    \end{definition}

\begin{theorem}\label{nontrivial_of_valExtension}
        \leanok
        \lean{nontrivial_of_valExtension}
        \uses{ValAlgEquiv.symm,Valuation.Nontrivial,Valuation.nontrivial_def,IsValExtension.val_map_eq_iff,IsValExtension.val_map_eq_one_iff,IsValExtension}
                No documentation.
    \end{theorem}

\begin{proof}
    \leanok
\end{proof}

\begin{theorem}\label{ringHomClass_eq_iff_valuationSubring}
        \leanok
        \lean{ringHomClass_eq_iff_valuationSubring}
        \uses{Valued.valuationSubring,ext}
                No documentation.
    \end{theorem}

\begin{proof}
    \leanok
\end{proof}

\begin{definition}\label{Valued.RingHom.restrictInteger}
        \leanok
        \lean{Valued.RingHom.restrictInteger}
        \uses{IsValExtension.val_map_le_one_iff,ValRingHom.id}
                No documentation.
    \end{definition}

\begin{theorem}\label{Valued.RingHom.restrictInteger_apply}
        \leanok
        \lean{Valued.RingHom.restrictInteger_apply}
        \uses{Valued.valuationSubring,Valued.RingHom.restrictInteger,IsValExtension}
                No documentation.
    \end{theorem}

\begin{proof}
    \leanok
\end{proof}

\begin{definition}\label{Valued.RingHom.restrictValuationSubring}
        \leanok
        \lean{Valued.RingHom.restrictValuationSubring}
        \uses{Valued.RingHom.restrictInteger}
                No documentation.
    \end{definition}

\begin{theorem}\label{Valued.RingHom.restrictValuationSubring_apply}
        \leanok
        \lean{Valued.RingHom.restrictValuationSubring_apply}
        \uses{Valued.RingHom.restrictValuationSubring}
                No documentation.
    \end{theorem}

\begin{proof}
    \leanok
\end{proof}

\begin{definition}\label{Valued.AlgHom.restrictValuationSubring}
        \leanok
        \lean{Valued.AlgHom.restrictValuationSubring}
        \uses{Valued.valuationSubring,ext,ValRingHom.id,Valued.RingHom.restrictValuationSubring,Valued.RingHom.restrictValuationSubring_apply,IsValExtension,IsValExtension.coe_algebraMap_valuationSubring}
                No documentation.
    \end{definition}

\begin{theorem}\label{Valued.AlgHom.restrictValuationSubring_apply}
        \leanok
        \lean{Valued.AlgHom.restrictValuationSubring_apply}
        \uses{Valued.decompositionGroup,Valued.valuationSubring,ValRingHom.comp,Valued.AlgHom.restrictValuationSubring,IsValExtension}
                No documentation.
    \end{theorem}

\begin{proof}
    \leanok
\end{proof}

\begin{definition}\label{Valued.DecompositionGroup.restrictValuationSubring}
        \leanok
        \lean{Valued.DecompositionGroup.restrictValuationSubring}
        \uses{Valued.decompositionGroup,ext,ValRingHom.comp,ValAlgEquiv.symm,ValAlgEquiv.apply_symm_apply,ValAlgEquiv.symm_apply_apply,Valued.AlgHom.restrictValuationSubring}
                No documentation.
    \end{definition}

\begin{theorem}\label{Valued.DecompositionGroup.restrictValuationSubring_apply}
        \leanok
        \lean{Valued.DecompositionGroup.restrictValuationSubring_apply}
        \uses{Valued.decompositionGroup,ValRingHom.comp,Valued.DecompositionGroup.restrictValuationSubring}
                No documentation.
    \end{theorem}

\begin{proof}
    \leanok
\end{proof}

\begin{theorem}\label{Valued.elem_decompositionGroup_eq_iff_ValuationSubring}
        \leanok
        \lean{Valued.elem_decompositionGroup_eq_iff_ValuationSubring}
        \uses{Valued.decompositionGroup,Valued.valuationSubring,ext,ValRingHom.comp,ringHomClass_eq_iff_valuationSubring,Valued.DecompositionGroup.restrictValuationSubring,Valued.DecompositionGroup.restrictValuationSubring_apply,IsValExtension}
                No documentation.
    \end{theorem}

\begin{proof}
    \leanok
\end{proof}

\begin{definition}\label{Valued.AlgEquiv.restrictValuationSubring}
        \lean{Valued.AlgEquiv.restrictValuationSubring}
        \uses{Valued.decompositionGroup,Valued.decompositionGroup_eq_top,eq_top,ValRingHom.comp,Valued.DecompositionGroup.restrictValuationSubring}
                No documentation.
    \end{definition}

\begin{theorem}\label{Valued.AlgEquiv.restrictValuationSubring_apply}
        \lean{Valued.AlgEquiv.restrictValuationSubring_apply}
        \uses{Valued.AlgEquiv.restrictValuationSubring}
                No documentation.
    \end{theorem}

\begin{theorem}\label{Valued.AlgEquiv.eq_iff_ValuationSubring}
        \lean{Valued.AlgEquiv.eq_iff_ValuationSubring}
        \uses{Valued.decompositionGroup,ext,ValRingHom.comp,Valued.elem_decompositionGroup_eq_iff_ValuationSubring,Valued.AlgEquiv.restrictValuationSubring}
                No documentation.
    \end{theorem}

\begin{theorem}\label{IsValExtension.val_map_le_iff}
        \leanok
        \lean{IsValExtension.val_map_le_iff}
        \uses{ValAlgebra.val_isEquiv_comap}
                No documentation.
    \end{theorem}

\begin{proof}
    \leanok
\end{proof}

\begin{theorem}\label{IsValExtension.val_map_lt_iff}
        \leanok
        \lean{IsValExtension.val_map_lt_iff}
        \uses{ValAlgebra.val_isEquiv_comap}
                No documentation.
    \end{theorem}

\begin{proof}
    \leanok
\end{proof}

\begin{theorem}\label{IsValExtension.val_map_eq_iff}
        \leanok
        \lean{IsValExtension.val_map_eq_iff}
        \uses{ValAlgebra.val_isEquiv_comap}
                No documentation.
    \end{theorem}

\begin{proof}
    \leanok
\end{proof}

\begin{theorem}\label{IsValExtension.val_map_le_one_iff}
        \leanok
        \lean{IsValExtension.val_map_le_one_iff}
        \uses{ValAlgebra.val_isEquiv_comap}
                No documentation.
    \end{theorem}

\begin{proof}
    \leanok
\end{proof}

\begin{theorem}\label{IsValExtension.val_map_lt_one_iff}
        \leanok
        \lean{IsValExtension.val_map_lt_one_iff}
        \uses{ValAlgebra.val_isEquiv_comap}
                No documentation.
    \end{theorem}

\begin{proof}
    \leanok
\end{proof}

\begin{theorem}\label{IsValExtension.val_map_eq_one_iff}
        \lean{IsValExtension.val_map_eq_one_iff}
        \uses{ValRingHom.id,ValAlgebra.val_isEquiv_comap,ValAlgebra.id.map_eq_id,ValAlgEquiv.mk',IsValExtension}
                No documentation.
    \end{theorem}

\begin{definition}\label{IsValExtension}
        \leanok
        \lean{IsValExtension}
        \uses{ValAlgebra.val_isEquiv_comap}
                The class `IsValExtension R A` states that the valuation of `A` is an extension of the valuation
on `R`. More precisely, the valuation on `R` is equivlent to the comap of the valuation on `A`.
    \end{definition}

\begin{definition}\label{IsValExtension.ofIntegerComap}
        \leanok
        \lean{IsValExtension.ofIntegerComap}
        \uses{DiscreteValuation.ofInt,ValAlgebra.val_isEquiv_comap,IsValExtension}
                No documentation.
    \end{definition}

\begin{definition}\label{IsValExtension.ofValuationSubringComap}
        \lean{IsValExtension.ofValuationSubringComap}
        \uses{DiscreteValuation.ofInt,IsValExtension.val_map_le_one_iff,ext,ValAlgEquiv.mk',IsValExtension,IsValExtension.ofIntegerComap}
                No documentation.
    \end{definition}

\begin{theorem}\label{IsValExtension.coe_algebraMap_integer}
        \leanok
        \lean{IsValExtension.coe_algebraMap_integer}
        \uses{IsValExtension}
                No documentation.
    \end{theorem}

\begin{proof}
    \leanok
\end{proof}

\begin{theorem}\label{IsValExtension.coe_algebraMap_valuationSubring}
        \lean{IsValExtension.coe_algebraMap_valuationSubring}
        \uses{Valued.valuationSubring,IsValExtension.val_map_eq_one_iff}
                No documentation.
    \end{theorem}

\begin{theorem}\label{IsValExtension.integerAlgebra_injective}
        \lean{IsValExtension.integerAlgebra_injective}
        \uses{Valued.valuationSubring,ext,injective,IsValExtension,IsValExtension.coe_algebraMap_valuationSubring}
                No documentation.
    \end{theorem}

